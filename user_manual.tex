%%%
%%% This user manual deliverable is based on monroe-deliverable.sty version 1.0
%%%


\documentclass[a4paper,10pt]{article}

\usepackage{monroe-deliverable}
\usepackage{appendix}
\usepackage{longtable}
\usepackage{mdwlist}		% Defines itemize*, enumerate* y description*, which have less vertical spacing.
\newcommand{\VerbatimFont}{\footnotesize}

%%% ALL figures MUST be inside the ``figures'' folder
%%% ``standard'' figures (like logos) are in the common ``templates/figures'' folder
\graphicspath{{figures/}{../templates/figures/}}
\DeclareGraphicsExtensions{.pdf,.jpg,.jpeg,.png}
\usepackage[binary-units]{siunitx}
\sisetup{detect-weight} % To detect bold fonts.
\def\defaultDigitGrouping{5}%So that it will always group, even with only 4 digits.
\sisetup{group-minimum-digits=\defaultDigitGrouping}

%%% 1st parameter is the acronym, 2nd is the full name
\setproject{MONROE}{Measuring Mobile Broadband Networks in Europe}

\setprojectnumber{644399}

\setECcall{H2020-ICT-11-2014}

%%% 1st parameter is the WP number, 2nd is the full name
\setworkpackage{5.2}{User Support}

%%% 1st mandatory parameter is the deliverable number, 2nd mandatory is the full name
%%% the optional parameter (between []) is a short name for headers, if needed
%%% use line breaks (\\) for the full name as needed to ensure proper formatting
\setdeliverable[MONROE Platform User manual]{User manual}{MONROE Platform User Manual}

%%% use \setprivatedeliverable for a confidential one
%\setprivatedeliverable
\setpublicdeliverable

%%% if this command is not used, then the default is: R (Report)
%%% 
%%% 1st parameter is the deliverable type code, 2nd is the desired name for this type;
%%% codes are as defined in table 3.1c of the proposal template:
%%%
%%%   r     = R: Document, report (excluding the periodic and final reports) 
%%%   dem   = DEM: Demonstrator, pilot, prototype, plan designs 
%%%   dec   = DEC: Websites, patents filing, press & media actions, videos, etc.
%%%   other = OTHER: Software, technical diagram, etc.
\setdeliverabletype{r}{Report}


%%% 1st parameter is the revision number, 2nd is the date
\setrevision{1.0}{\today}


%%% 1st parameter is the path to the file with the project logo, 2nd is the scale factor
\setprojectlogo{monroe_logo}{0.9}


%%% 1st parameter is the path to the file with the H2020 / EU logo, 2nd is the scale factor
\setEClogo{EC-H2020}{0.3}


%%% use line breaks (\newline, NOT \\) as needed to ensure proper formatting of the author list
%%% the optional parameter (between []) is an alternative to "Contributor(s)", if desired
\setdeliverableauthors{Miguel Pe\'{o}n-Quir\'{o}s, Thomas Hirsch}


%%% use line breaks (\newline, NOT \\) as needed to ensure proper formatting of the editor list
%%% can be omitted if desired
\setdeliverableeditors{\"Ozg\"u Alay, Vincenzo Mancuso}


%-------------------------------------------------------------%
%-------------  Custom commands and environments  ------------%
%-------------------------------------------------------------%

%%% add your own macro (& color) to have your own inline color comments
%%% other (readable) colors: orange, blue, teal, violet, purple, brown, olive, magenta, cyan
\newcommand{\OA}[1]{\textcolor{red}{{\sf OA: #1}}}
\newcommand{\MPQ}[1]{\textcolor{blue}{{\sf MPQ: #1}}}
\newcommand{\todo}[1]{\textcolor{red}{TODO: #1}\PackageWarning{TODO:}{#1!}}  
\newcommand{\monroe}{MONROE}
% Style for identifiers such as "main()" or file names:
\newcommand{\identifier}[1]{{\texttt{\small{#1}}}}




%------------------------------------------------------------%
%-------------  the actual document begins HERE  ------------%
%------------------------------------------------------------%

\begin{document}


%%% cover page
\makecoverpage
\thispagestyle{empty}

\newpage


%%% second cover page

\deliverableabstract{This document describes the processes that \monroe{} experimenters need to follow to create, run, monitor and collect results from their experiments.}

%%% if keywords are not given, they are automatically not added to the second cover page
%\deliverablekeywords{}

\makesecondcoverpage

\vspace*{\fill}

\begin{center}

%%% list of partners and their acronyms added by hand here
\begin{tabular*}{12cm}{p{9.5cm}@{\hspace{0pt}}c}
\toprule
\textbf{Participant organisation name} & \textbf{Short name} \\
\midrule
SIMULA RESEARCH LABORATORY AS \emph{(Coordinator)} & SRL \\
CELERWAY COMMUNICATION AS & Celerway \\
TELENOR ASA & Telenor \\
NETTET SVERIGE AB & NET1 \\
NEXTWORKS & NXW \\
FUNDACION IMDEA NETWORKS & IMDEA \\
KARLSTADS UNIVERSITET & KaU \\
POLITECNICO DI TORINO & POLITO \\
\bottomrule
\end{tabular*}
\end{center}

\vspace*{\fill}

\newpage


%%% ToC, may be removed for very short documents
\tableofcontents
\newpage



%------------------------------------------------------------%

%\section*{Executive summary}
%
%\OA{this section may be present in very long documents, else remove}
%
%\newpage

%------------------------------------------------------------%

\section{Introduction}
\label{sec:intro}

The purpose of this document is to guide \monroe{} experimenters through the process of creating, running and monitoring their experiments, and the subsequent collection of results.
It first explains how the experiments must be prepared inside Docker containers, the testing process they must undergo before they can be deployed into \monroe{}'s nodes, and how they must be uploaded to a repository for deployment into the nodes.
Then, it explains the basics of the web interface that allows for the provision of resources and the scheduling of experiment executions.
Finally, it shows how the experiment results can be retrieved either directly from the experiment itself or from the repository provided by \monroe{}.

\subsection{Overview of the node configuration}

\monroe{} nodes have been designed to have minimal impact on the experiments that run on them.
Therefore, only one experiment can run at a given time in a node.
Although the experiments are executed inside a Docker container, they have no quotas on CPU or memory usage, subject only to available node resources.
Container image size and temporary storage in the node may be restricted, though.

Every \monroe{} node runs, in addition to user experiments, the following background processes:

\begin{itemize}
	\item The experiment scheduler, which arbitrates the execution of user experiments in the node. The scheduler runs permanently in the background and contacts periodically the scheduling server, sending ``heartbeats'' and checking for new schedules for the node.
	When an experiment is not running, the scheduler may start the deployment of the containers for one or several experiments scheduled to be run in the immediate future, so that they are prepared on advance.
	The scheduler checks the duration of the slot assigned to an experiment; if the experiment does not finish on time, it stops the whole container.
	\item Synchronization (rsync) services to copy data files to the \monroe{} repository. This service copies user experiment results, the data collected by passive experiments and assorted metadata measurements. It runs continuously, transferring files to the server as they appear in the corresponding folders. This service uses the management interface, which is different of the interfaces available for the experiments. However, the management interface may share in some cases the same subscriber contract with one of the experiment interfaces; operators might restrict the total bandwidth available for all the SIMs linked to the same contract. Additionally, two modems (management plus experiment) using the same operator antenna may somehow affect the bandwidth available for the experiment. Therefore, experimenters should be aware of the small amount of data that can be transferred by this service in parallel to their experiments.
	\item Several systems run continuously in the background gathering information on the status of diverse components. Examples include a service to read the signal strength and network configuration of each of the experiment modems, the GPS data and various node parameters such as CPU load, memory usage or CPU temperature. These services run continuously in the background with a frequency that varies from one second up to several minutes. Although their impact on user experiments should be minimal, their existence must be known by the experimenters.
	\item In addition to the services that gather metadata, \monroe{} nodes keep several containers active all the time. These containers run experiments that are deemed basic for the \monroe{} platform and include:
	\begin{itemize}
		\item A ping experiment. Container number \num{1} executes continuously an ICMP ping operation to a fixed external server (currently, Google's DNS at 8.8.8.8). The RTT values are collected and transferred to the servers. The ping experiment runs continuously with a frequency of one second, for every interface.
		\item A bandwidth measurement test, which periodically downloads an object using the HTTP protocol to measure the achievable bandwidth. The test runs on each interface. The periodicity of this experiment and whether it can be run while user experiments are being executed are yet to be decided.
		\item A container that periodically executes a full paris-traceroute to several popular websites and records information about all the intermmediate hops. This container will in principle be run three times per day, but the interactions with user experiments are yet to be determined.
		\item A container that runs Tstat, the passive monitoring tool that collects, for each interface, detailed flow level statistics. The Tstat container generates no traffic; flow level data is synchronized to the \monroe{} repository using the standard synchronization process described above.
	\end{itemize}
\end{itemize}

\subsection{Overview of the experimental workflow}

Experiments conducted in the \monroe{} platform follow the workflow shown in Figure~\ref{fig:ExperimentWorkflow}.
They consist of three phases: Experiment design, testing and experimentation.
During the experiment design phase, the experiment goals and properties are defined and the container required to deploy it in \monroe{} nodes is configured.
During the testing phase, the container is executed on nodes specifically devoted to testing new experiments.
If the experiment passes all the safety and behavior tests, a \monroe{} manager will digitally sign the container image.
Signed containers cannot be further modified without running again through the testing phase.
Finaly, the experimenter is free to schedule the experiment container on any nodes, subject to the specific quotas assigned to their project.

\begin{figure}[tp]
	\centering
	\includegraphics[width=1.0\textwidth,trim={7cm 7.5cm 7cm 5cm}]{ExperimentWorkflow.pdf}
	\caption{Experimental workflow.}
	\label{fig:ExperimentWorkflow}
\end{figure}


%------------------------------------------------------------%

\section{Experiment preparation}
\label{sec:experimentPreparation}

%Abcd.

\subsection{General experiment notes}
\label{subsec:generalExperimentNotes}

MONROE experiments run inside a Docker container under the root user of the container.
Therefore, the experimenters can design any kind of experiment within the security restrictions of the platform, including the configuration of routing tables, stopping or starting interfaces and executing any kind of applications.
We assume the reader is familiar with the Docker technology.
If not, we strongly suggest getting used to it by accessing the documentation at \url{https://docs.docker.com/engine/understanding-docker/}~.

Creating and using containers is a two-step process.
At design time, the experimenters create the image for the container in their local machine using a container-creation script.
If necessary, they can install new packages (e.g., via apt-get) or copy libraries.
The docker tools read the script and create the final image for the experiment, which will then have to be uploaded to a repository.
At run-time, the nodes retrieve the container image from the repository and start it as scheduled.

During execution, the experiment should not install additional applications or download any data that is not part of the experiment itself (e.g., if the experiment uploads a file to a server to test upstream speed, either include the file to be uploaded in the container at design time or create it locally).

\emph{$\Rightarrow$~Experiments will under no circumstances allow direct ssh access to the node or any other form of running interactive commands from outside the container that can pose a security risk for the platform.~$\Leftarrow$}

\subsection{Container preparation}
\label{subsec:containerPreparation}

\monroe{} experiments are deployed in Docker containers (\url{https://www.docker.com/}).
Preparing a new container from \monroe{}'s base image is an easy process:

\begin{enumerate}
	\item Install Docker in your machine. Do it preferably downloading the installation script  from the web page, rather than through a package manager such as apt-get:
		{\VerbatimFont\begin{verbatim}
			$ wget https://get.docker.com -O install.sh
			$ chmod u+x install.sh
			$ ./install.sh
		\end{verbatim}}
		You will have to run docker as root unless you add yourself to the docker group.

		Mac users: Download and install ``Docker for MAC'' (\url{https://www.docker.com/products/docker#/mac}) or the ``Docker Toolbox'' (\url{https://docs.docker.com/toolbox/overview/}), according to your OS version.
	\item Test Docker installation with the `Hello world!' example:
		{\VerbatimFont\begin{verbatim}
			$ sudo docker run hello-world
			Unable to find image 'hello-world:latest' locally
			latest: Pulling from library/hello-world
			03f4658f8b78: Pull complete
			a3ed95caeb02: Pull complete
			Digest: sha256:xxxxxxxxxxxxxxxxxxxxxxxx
			Status: Downloaded newer image for hello-world:latest
		\end{verbatim}}
		If everything has worked correctly up to here, you will see a welcome message similar to the following:
		{\VerbatimFont\begin{verbatim}
			Hello from Docker.
			This message shows that your installation appears to be working correctly.
			...
		\end{verbatim}}
		You can check which images are locally installed with:
		{\VerbatimFont\begin{verbatim}
			$sudo docker images
			REPOSITORY    TAG      IMAGE ID       CREATED        SIZE
			hello-world   latest   690ed74de00f   4 months ago   960 B
		\end{verbatim}}
  \item Now you are ready to download the \monroe{} base image:
		{\VerbatimFont\begin{verbatim}
			git clone https://github.com/MONROE-PROJECT/Experiments.git
		\end{verbatim}}
		This will fetch the repository with \monroe{}'s example containers.
	\item Head to \identifier{Experiments/experiments/template/}.
		Here, you will find the required files to prepare your image based on \monroe{}'s base.
		You should care about four things: a) The contents of the \identifier{files/} folder, b) the \identifier{build.sh} file, c) the \identifier{push.sh} file and d) the \identifier{template.docker} script file that describes how to create your container.
		In the directory \identifier{files/} you can put all the files that are part of your experiment.
		As a simple example, we can use the following script:
		{\VerbatimFont\begin{verbatim}
			$vi files/myscript.sh
			  #!/bin/bash
			  ls -lah > /monroe/results/listing.txt
		\end{verbatim}}
		Any files that your experiment creates in \identifier{/monroe/results} will be retrieved after its completion and delivered to the repository, where you will be able to finally retrieve them.
		\emph{Writes to any other part of the filesystem will be lost once the experiment is finished.}
		In periodic schedules, no data will survive from one execution to the next (i.e., the container is loaded fresh before each execution).
		If result persistence is needed, the experimenter will have to supply it by downloading the needed files from the network during the experiment itself.
	\item You should not need to modify the \identifier{build.sh} file. The name of the container is the name of the current directory, and it must match the name of the \identifier{.docker} file (e.g., \identifier{template.docker} as we are in a folder named \identifier{template/}).
	
	\item The file \identifier{template.docker} is the script used to build your container.
		You can modify it to:
		\begin{itemize}
			\item Define the entry point of your experiment (``ENTRYPOINT'').
			\item Change the base image of the container, e.g., \identifier{monroe/base}.
			\item Install additional packages or libraries.
		\end{itemize}
		For example:
		{\VerbatimFont\begin{verbatim}
			FROM monroe/base
		
			MAINTAINER your-email-address
		
			COPY files/* /opt/monroe/
		
			#Default cmd to run.
			ENTRYPOINT ["dumb-init", "--", "/bin/bash", "/opt/monroe/myscript.sh"]
		\end{verbatim}}
	
		This example will copy the files in the \identifier{files/} directory to the one you specify \emph{inside} the docker container (e.g., \identifier{/opt/monroe}).
		
		TIP: If you need to install additional packages in the container, be sure to clean any temporary files from the image. Also, notice that the Docker creation script analyses the contents of the container filesystem after every line in the .docker script is executed. That means that, even if you delete files at the end, Docker will create intermediate ``layers'' that will be downloaded and applied sequentially to build the final image of your container. Consider instead using one-liners such as the following:
		{\VerbatimFont\begin{verbatim}
		    RUN apt-get update && apt-get install -y vim && apt-get clean
		\end{verbatim}}
		This will ensure that the files are deleted before Docker analyses the filesystem.
	\item Modify the file \identifier{push.sh} to reflect the name of your repository:
		{\VerbatimFont\begin{verbatim}
			#!/bin/bash
			DIR="$( cd "$( dirname "${BASH_SOURCE[0]}" )" && pwd )"
			
			CONTAINER=${DIR##*/}
			
			CONTAINERTAG=myuser/myrepo # --> Modify to your own dockerhub user/repo
			
			docker login && docker tag ${CONTAINER} ${CONTAINERTAG} && docker push ${CONTAINERTAG} && \
			    echo "Finished uploading ${CONTAINERTAG}"					
		\end{verbatim}}
		$\rightarrow$During the prototype phase of the system, please follow these additional steps to make your container available:
		\begin{itemize}
			\item Create an account at Docker Hub.
			\item Create a private repository (you can create one container as private; no limits for public ones).
			\item In your development machine, run: \identifier{docker login}. It will ask you for your credentials.
		\end{itemize}
	\item After populating the \identifier{files/} directory, modifying the .docker file and updating the \identifier{push.sh} file, you are ready to create the image:
		{\VerbatimFont\begin{verbatim}
			$sudo ./build.sh
			Using default tag: latest
			latest: Pulling from monroe/base
			Digest: sha256:6df1195a3cc3da2bfe70663157fddc42e174ec88761ead7c9a3af591e80ebbd5
			Status: Image is up to date for monroe/base:latest
			Sending build context to Docker daemon 11.26 kB
			Step 1 : FROM monroe/base
			---> d1b4f4baa60d
			Step 2 : MAINTAINER mikepeon@imdea.org
			---> Using cache
			---> 0b05b5c453c7
			Step 3 : COPY files/* /opt/monroe/
			---> acc2df443070
			Removing intermediate container 66a666516a27
			Step 4 : ENTRYPOINT dumb-init -- /bin/bash /opt/monroe/myscript.sh
			---> Running in f4b7a1ee804a
			---> 096c7a56ff1c
			Removing intermediate container f4b7a1ee804a
			Successfully built 096c7a56ff1c
			Finished building template
		\end{verbatim}}
	\item Test that your new docker container is available:
		{\VerbatimFont\begin{verbatim}
			$sudo docker images
			REPOSITORY                                   TAG    IMAGE ID      CREATED         SIZE
			hello-world                                  latest 690ed74de00f  4 months ago    960 B
			your_docker_account/your_experiment          latest xxxxxxxxxxxx  32 seconds ago  626.6 MB
			monroe/base                                  latest xxxxxxxxxxxx  12 days ago     626.6 MB
		\end{verbatim}}
		Exact image ids and sizes will vary.
	\item Push the container image to the repository:
		{\VerbatimFont\begin{verbatim}
			$ sudo ./push.sh
			Username (your-Docker-user-name):
			Password: (type your DockerHub password)
			WARNING: login credentials saved in /home/your-username/.docker/config.json
			Login Succeeded
			The push refers to a repository [docker.io/mikepeon/template]
			5f339bfdaae2: Pushed
			486ab26686cc: Layer already exists
			034f70c0d9cd: Layer already exists
			86b5acd8772a: Layer already exists
			f03317610243: Layer already exists
			50f6c1bd7ce6: Layer already exists
			aec5953bffa2: Layer already exists
			507169b05eea: Layer already exists
			5d799297d10c: Layer already exists
			759d76df9ac7: Layer already exists
			5f70bf18a086: Layer already exists
			12e469267d21: Layer already exists
			latest: digest: sha256:c855de65307191b4832b2ec60a4401c1b63424827c29149703c5d7ef07b519f7 size: 3001
			Finished uploading your-username/template
		\end{verbatim}}
	\item You can now test that your image runs correctly, even on your own PC (if the experiment logic and resource demands allow for it).
		{\VerbatimFont\begin{verbatim}
			$mkdir /run/shm/myresults
			$sudo docker run -v /run/shm/myresults:/monroe/results your_docker_account/your_experiment
			    --> The output of your experiment will be in /run/shm/myresults/listing.txt
		\end{verbatim}}
		The docker command line allows you to specify a mapping between a directory inside the docker image and one in the host system.
		In this case, we have mapped \identifier{/monroe/results} from the container to \identifier{/run/shm/my\-results}.
		This is useful if you are running the container locally in a normal PC for debugging purposes.
		IMPORTANT: This process shows how to build and run a container \emph{locally} in your workstation.
		However, experimenters do not have direct access to the \monroe{} nodes.
		Therefore, to execute your experiment \emph{in} a \monroe{} node, you will follow the process just up to the \identifier{sudo ./push.sh} step and then use the web interface to upload and schedule the container into the nodes.
\end{enumerate}

You may check the contents of \identifier{experiments/*} for more useful examples.

The following is a list of useful common Docker commands:
\begin{itemize*}
	\item To list installed/built images (and get their ids):
	\begin{verbatim}docker images\end{verbatim}
	\item To list running containers and get their tags:
	\begin{verbatim}docker ps\end{verbatim}
	\item To stop running containers:
	\begin{verbatim}docker kill container-tag\end{verbatim}
	\item To delete images:
	\begin{verbatim}docker rm -f image_id\end{verbatim}
	\item To retrieve the latest version of an image (e.g., \identifier{monroe/base}):
	\begin{verbatim}docker pull monroe/base\end{verbatim}
	\item To attach to a running container and get an interactive shell:
	\begin{verbatim}docker exec -i -t container-tag bash\end{verbatim}
\end{itemize*}

\subsubsection{Package and tool installation}
\label{subsubsec:packagesInstallation}

If you have to install extra packages, libraries or tools, do it from the \identifier{my\_experiment.docker} file.
You should never pull repositories or download libraries from inside your experiment as this will count against your data quota (and execution slot) for every instance of your experiment.
Instead, modify the container configuration file as in the following example:
{\VerbatimFont\begin{verbatim}
			FROM monroe/base
			
			MAINTAINER your-email-address
			
			RUN apt-get update && apt-get install -y \
			    python \
			    python-pip \
			    traceroute \
			    && apt-get clean
			RUN pip install pygame
			
			RUN mkdir -p /opt/yourname
			COPY files/* /opt/yourname/
			
			#Default cmd to run
			ENTRYPOINT ["dumb-init", "--", "/bin/bash", "/opt/yourname/myscript.sh"]
\end{verbatim}}

You may also download any files using \identifier{wget}, but you may simply put them in the \identifier{files/} folder as well.
Remember, this happens during container creation on your PC, \emph{not} during experiment execution on the nodes.

If you find the need for big libraries that you think should go into the base image, please contact \monroe{}'s administrators.

TIP: The easiest way to find out which packages and versions are available in the \monroe{} base image is to create a simple container and run an interactive batch session inside it in your workstation.
For example, assuming that you have a basic container that simply waits when run, you may follow the following steps:
{\VerbatimFont\begin{verbatim}
	mkdir /run/shm/myresults
	docker run -v /run/shm/myresults:/monroe/results repository/your_container &
	docker ps --> Look for the tag of your running container
	docker exec -i -t container_tag bash
	--> Here you are inside the container
	dpkg -l > /monroe/results/package-listing.txt
	exit
	--> You'll find the output at /run/shm/myresults/package-listing.txt
\end{verbatim}}
For easier reference, Table~\ref{tab:installedPackages} in Appendix~\ref{app:installedPackages} gives a detailed listing of packages available in \identifier{monroe/base} at the time of writing this text.

\subsection{Optional interactive debugging in \monroe{} nodes}
To make the process of debugging experiments in the nodes, a small number of development nodes will be supplied.
Experimenters will be able to schedule their containers to one of these nodes (even before getting their container signed) and, once the container is started, connect to it through an inverse SSH tunnel.
Then, they will be able to interactively modify their scripts or applications and execute them until the allocated slot expires.

The exact technical procedures and reservation policies are still to be defined by the \monroe{} consortium.

\subsection{Mandatory testing process}
\label{subsec:testingProcess}
Every experiment submitted to the \monroe{} testbed \emph{must} first pass through a testing process to receive manual approval by a \monroe{} administrator.
To submit your experiment for testing, you have to use the web interface specifying ``testing'' as the required node type.

\subsection{Deployment}
\label{subsec:deployment}
\monroe{}'s scheduling system will automatically deploy experiments to the nodes before their execution time.
The nodes will fetch the container image from the Docker repository, and the size of the download will be accounted in your data quota.
Notice that in the case of periodic experiments, each time an experiment is run, the Docker container will possibly be re-downloaded and its costs will be accounted in your quota.

%------------------------------------------------------------%

\section{Resource allocation, and experiment scheduling and monitoring}
\label{sec:allocSchedMonitor}

Once a experiment is configured as a Docker container, it can be scheduled multiple times under different conditions using the user client web located at \url{https://www.monroe-system.eu}~.

\subsection{User login and certificates}
\label{subsec:login}
User identification in \monroe{} is achieved through client certificates.
Every experimenter has their own certificate compatible with the FED4FIRE\footnote{\url{http://www.fed4fire.eu/}} federation.
User certificates are issued by iMinds through the following URL: \url{https://authority.ilabt.iminds.be/}.
New users must create a new account (``sign up'').
Be sure to select the option ``Join Existing Project'' and type the name ``Monroe'' in the project name field (Figure~\ref{fig:iMindsCreateAccount}).
The authorization process involves a manual verification step by one of the \monroe{} administrators, so it will probably take one or two days.

\begin{figure}[tp]
	\centering
	\includegraphics[width=0.5\textwidth]{iMindsCreateAccount2.png}
	\caption{iMinds registration page to obtain FED4FIRE-compatible certificates for use with the \monroe{} platform.}
	\label{fig:iMindsCreateAccount}
\end{figure}

Please, notice that the current policy for \monroe{} is to use one user certificate per project, shared between all the experimenters belonging to that project.

Once the identity of the experimenter is approved, they will receive an email to download the certificate files.
These files must be installed in the experimenter browser.
After that, the user should be able to access the user web directly.
Upon request of the main (index.html) file, the browser will contact \monroe{} servers to verify that the user credentials are correct.
In the case of any problems, the user will be presented with instructions on how to obtain a certificate.
If the client certificate is verified successfully, they will be automatically redirected to the listing of their experiments.

\textbf{NOTE:} User certificates are manually activated in the scheduling software.
To use your certificate, please send its ``ssl\_id'' to one of the \monroe{} administrators.
You may find it in the screen after pressing the ``Try me'' button, once the certificate is correctly installed in your browser:
{\VerbatimFont\begin{verbatim}
{
"fingerprint": "c79f1967aea17811a1ebed39b7d718430904338a",
"user": {
    "id": 3,
    "name": "MONROE Test admin",
    "quota_data": 50000000000,
    "quota_storage": 500000000,
    "quota_time": 500000000,
    "role": "admin",
\end{verbatim}
\color{red}
\vspace{-0.7cm}
\begin{verbatim}
    "ssl_id": "c79f1967aea17811a1ebed39b7d718430904338a"
\end{verbatim}
\color{black}
\vspace{-0.7cm}
\begin{verbatim}
},
"verified": "SUCCESS"
}
\end{verbatim}}

~\\$\Rightarrow$~We have identified some common issues that are not yet solved. Below are some workarounds:
\begin{itemize}
	\item For the first login, you may be asked for your user certificate and then your browser may show a security warning. This is due to the use of a self-signed server certificate. Please ask your browser to proceed. Then, you will probably see an error page from \monroe{}. Please, click the red button labeled ``Try me'' and check that you get a successful data output. Finally, please proceed again to the main page of the project. From that point, you should be able to access the system without further problems in future sessions. (Pointers on how to simplify this issue are welcome!)
	\item Firefox on OSX has an issue with CORS headers. Although the web and scheduling servers are running now on the same machine, you may still encounter this problem.
\end{itemize}

\subsubsection{Installation of user certificates in Chrome}
\label{subsec:userCertsInstallWinChrome}

This section explains how to install the FED4FIRE-compatible user certificates used by the \monroe{} platform in Google Chrome for Windows.
The procedure for other browsers and platforms should be similar.

\begin{enumerate}
	\item Go to your browser settings page:\\
	\begin{center}\includegraphics[width=0.7\textwidth]{InstallCert01.png}\end{center}
	
	\item Display the advanced configuration settings:\\
	\begin{center}\includegraphics[width=0.7\textwidth]{InstallCert02.png}\end{center}
	
	\item Go to the section labeled ``HTTPS/SSL'' and click the button ``Manage certificates...'':\\
	\begin{center}\includegraphics[width=0.7\textwidth]{InstallCert03.png}\end{center}
	
	\item The dialog box for managing your certificates will be displayed. Press the button ``Import...'' to import your certificate:\\
	\begin{center}\includegraphics[width=0.5\textwidth]{InstallCert04.png}\end{center}
	
	\item In the new dialog, click ``Next'':\\
	\begin{center}\includegraphics[width=0.5\textwidth]{InstallCert05.png}\end{center}
	
	\item In the file-selection dialog that appears next, change the file type from ``X.509 Certificate (*,cer;*.crt)'' to ``Personal Information Exchange'':\\
	\begin{center}\includegraphics[width=0.6\textwidth]{InstallCert06.png}\end{center}
	
	\item And select the file containing your certificate:\\
	\begin{center}\includegraphics[width=0.6\textwidth]{InstallCert07.png}\end{center}
	
	\item In the next dialog box, enter your certificate password:\\
	\begin{center}\centering\includegraphics[width=0.5\textwidth]{InstallCert08.png}\end{center}
	
	\item If the import is successful, your certificate will be imported to your ``Personal Store'' and you will be able to access the \monroe{} user interface by selecting it when prompted by your browser. Notice that you may still get a warning about the validity of the server certificate.
	
\end{enumerate}


\subsection{Resource allocation}
\label{subsec:resourceAllocation}
New instances of experiments are created, assigned resources and scheduled under the tab ``New.''
Here, the user will be presented with a page similar to Figure~\ref{fig:newExperimentBlank}.

\begin{figure}[tp]
	\centering
	\includegraphics[width=1.0\textwidth]{NewExperiment_blank.png}
	\caption{Example for the creation of a new experiment.}
	\label{fig:newExperimentBlank}
\end{figure}

To create a new experiment, the user must specify at least the following parameters:
\begin{description}
	\item [Name:] A representative experiment description.
	\item [Script:] A Docker hub path for the experiment container. In the previous example, it would be \identifier{your\_\allowbreak docker\_\allowbreak account/my\_\allowbreak experiment.}
	\item [Number of nodes:] The number of nodes that will execute the experiment.
	\item [Duration:] Length of the experiment execution, in seconds (excluding the time required to deploy the container).
	The node will kill the experiment after this time. The minimum slot that can be reserved is $\SI{5}{\minute}$ and the maximum, $\SI{24}{\hour}$.
\end{description}

If the starting date is fixed, the user can introduce it in the field ``Start.''
All dates are introduced as UTC times; the interface presents alongside the corresponding local time for the user's browser.
The scheduler will then try to satisfy the requirements.

Alternatively, if the starting date is not relevant, the user may leave this field empty and press the button ``Check availability'' to check the earliest available slot (please, add at least ten minutes to the proposed time to allow for container deployment into the nodes).
If the user just wants to submit the experiment as soon as possible, they can just mark the option ``As soon as possible'' and leave the other fields empty when pressing the ``Submit'' button.

Additionally, the user may specify the following restrictions (Figure~\ref{fig:newExperimentFilters}):
\begin{description}
	\item [Country filter:] The user may select nodes located in one or several countries, or they may choose to use nodes from any country indistinctly.
	\item [Node type:] %Available node types are static or mobile (e.g., in buses or trains).
	%Some mobile nodes may have restricted availability.
	%The ``testing'' nodes are reserved for experiments that must still be sanctioned by a \monroe{} administrator.
	Currently available node types are deployed or testing.
	The ``testing'' nodes are reserved for experiments that must still be verified by a \monroe{} administrator.
	\item [Node IDs:] If the experimenter wants to use a set of specific nodes, for example, to repeat one experiment under the very same conditions, it is possible to introduce a comma-separated list of required nodes, instead of accepting any available ones.
	\item [Required interfaces and active-data quota:] By default, experiments do not have access to the node network interfaces.
	The experimenter must explicitly select those interfaces that they want to be available for their container.
	The selected interfaces will be mapped one-to-one from the physical ones.
	The experimenter must also specify the active-data quota for each interface, that is, the maximum amount of data that each interface can use.
	The scheduler multiplies this value by the number of interfaces used and checks the result against the quota available for the user.
	\item [Log files quota:] The user may want to place an estimate on the maximum amount of data that may be generated as result files in \identifier{/monroe/results}.
	This is important because the size of the results is also counted against the user quota.
	
	\item [Recurrence:] \monroe{}'s scheduler allows to specify experiments that need to be repeated periodically.
	In that case, the user has to specify the repetition period ($\ge \SI{3600}{\second}$) and the final stopping date.
	The scheduler will treat each repetition as a different experiment and will try to satisfy the requirements for each of them consecutively.
	However, the operation is atomic:
	Either all the repetitions are scheduled, or none are.
\end{description}

\begin{figure}[tp]
	\centering
	\includegraphics[width=0.9\textwidth]{NewExperimentFilters.png}
	\caption{Filters for node selection.}
	\label{fig:newExperimentFilters}
\end{figure}

\subsection{Experiment scheduling}
\label{subsec:experimentScheduling}

When all the requirements are specified, the user needs to click the ``Submit experiment'' button to submit to the scheduler.
The experiments must respect several restrictions to be successfully scheduled:
\begin{itemize}
	\item The starting time must be at least \SI{10}{\minute} in the future, to allow time for container deployment.
	\item No experiment can be scheduled more than one month in advance.
	\item Periodic experiments must have a period greater than \SI{3600}{\second}.
	The finishing time must also obey the previous rule, that is, the last experiment instance in the recurrence must be scheduled in less than a month from the current time.
	\item No experiment (or instance in a series) can last more than one day.
	\item If a list of specific nodes and a starting date are given, the scheduler may be unable to grant the required resources.
%	TODO: Find a way to ease the problem of finding a slot, with perhaps a time-line of resource availability.
\end{itemize}

\subsubsection{Checking availability}
If the exact starting time is not relevant, the user can press the ``Check availability'' button.
If the requirements can be satisfied, a message explaining when the experiment might be started will be displayed.
Additionally, it will also inform of the maximum number of nodes that can be used during this period, and the maximum ending time.
With these data, the experimenter may decide to increase the number of nodes that run the experiment, or increase its duration until the time that the scheduler is likely being able to grant.
Figure~\ref{fig:newExperimentCheckAvailability} shows the answer of the scheduler for an availability query.

\begin{figure}[tp]
	\centering
	\includegraphics[width=1.0\textwidth]{NewExperimentCheckAvailability.png}
	\caption{The scheduler may supply hints on the scheduling availability, including the earliest starting date that is possible, the end of the availability period for the required resources and the maximum number of nodes, with the specified requirements, that the experiment could reserve. In this example, the experiment can start on ``2016-05-18 14:45:00 UTC'' and can last until ``2016-06-18 14:44:00 UTC.'' The experiment can be scheduled with up to 109 nodes during this period.}
	\label{fig:newExperimentCheckAvailability}
\end{figure}


\subsection{Experiment monitoring}
\label{subsec:experimentMonitoring}

Once an experiment is successfully submitted, the user can check its progress under the ``Status'' tab.
Figure~\ref{fig:ExperimentsSummary} shows an example of a list of experiments.

\begin{figure}[tp]
	\centering
	\includegraphics[width=0.925\textwidth]{ExperimentSummary.png}
	\caption{List of user experiments.}
	\label{fig:ExperimentsSummary}
\end{figure}

All the active (i.e., not completed) experiments for the user are shown.
Experiments that have not yet been started can be canceled and deleted.
However, the scheduler will try to stop experiments that have already started, but they will not be deleted from the list.

Clicking on any experiment displays the details for its individual schedules at the bottom of the page.
There, the number of schedules that are defined but not yet deployed, the ones that are deployed and ready to be started, the ones that are currently running, etc., is summarized.
One line is presented for each individual schedule on each \monroe{} node.
The following list explains the meaning of the states in which an individual task may be:
\begin{description*}
	\item [\textbf{Finished}:] The task was correctly executed and it finished on its own before consuming the complete time slot. 
	\item [\textbf{Stopped}:] The task was correctly executed, but it was stopped by the scheduler at the end of the execution slot (see note below).
	\item [\textbf{Failed}:] The task stopped abnormally.
	\item [\textbf{Canceled}:] The task was cancelled by the user before being started.
	\item [\textbf{Aborted}:] The task was aborted by the user after being started.
	\item [\textbf{Defined}:] The task is known to the scheduler.
	\item [\textbf{Deployed}:] The corresponding node has already deployed the container and is waiting for its starting time.
	\item [\textbf{Started}:] The container is running in the designated node. The ``download'' link for the task results is already available.
\end{description*}

Some experiments may be designed to finish after completion.
For those ones, the correct finishing state is ``Finished.''
If they are stopped by the scheduler, they probably exceeded the execution time foreseen by the experimenter.
However, some experiments may be designed to run continuously for a period of time.
In those cases, the ``Stopped'' state could actually be the correct ending state as intended by the experimenter.

%------------------------------------------------------------%

\section{Retrieval of experiment results from \monroe{}'s repository}
\label{sec:resultRetrieval}

Any files written during the experiment to the \identifier{/monroe/results} directory will be synchronized to the experiment repository.
This operation happens continuously during experiment execution and then upon its finalization.
Therefore, it is advisable that only final files ready to be transferred are copied to that location to avoid the system to sync temporary files and consume your quota or produce invalid results.

The result files can be accessed through the user interface:
For experiments that have already been started, the interface presents a link under the column ``Results'' that redirects the user to the HTTP folder (Figure~\ref{fig:ResultsRetrieval}) that contains the files that have already been synchronized from the node where the experiment runs to the repository.
In this way, the experimenter can retrieve result files even for partial experiments that fail or are canceled.

\begin{figure}[tp]
	\centering
	\includegraphics[width=0.7\textwidth]{ResultsRetrieval.png}
	\caption{Folder containing the results of an individual schedule, transferred to \monroe{}'s servers.}
	\label{fig:ResultsRetrieval}
\end{figure}

In addition, the experiment may use the network functionalities to communicate with any outside servers as needed (e.g., scp some files to an external server).

%------------------------------------------------------------%

\section{Run-time considerations for experimenters}
\label{sec:experimentRuntime}

This section discusses several considerations that experimenters must take into account when designing and running their experiments on the \monroe{} platform.

\subsection{Communication during the experiment}
\label{subsec:communicationDuringExperiment}

During execution, the experiment is free to establish any network communications through the available interfaces.
The user can choose to bind explicitly from each command or application to a specific interface, or they may define default routes during the experiment:
{\VerbatimFont\begin{verbatim}
	route add default gw 172.16.0.1 eth0
\end{verbatim}}

\subsection{Interface naming and binding, and default route}
\label{subsec:interfaceNaming}

$\Rightarrow$~\emph{This section contains technical details that are currently subject to evaluation by the \monroe{} consortium. Comments from experimenters on this section are welcome.}~$\Leftarrow$\\

Experiments running in \monroe{} nodes have access to several network interfaces.
By default, that is, if the experiment does not take any special configuration actions, the default route will be configured to one of the mobile broadband interfaces, if available.
However, experimenters have the possibility of explicitly binding external tools or their programs to specific interfaces.
For example, the standard ``ping'' tool can be forced to use an specific interface with the following command:
{\VerbatimFont
\begin{verbatim}
	ping -I eth0 host_name
\end{verbatim}}

To offer a consistent view of the platform resources, whereas allowing flexibility for future changes in the platform configuration, the following naming scheme is used for each of the interfaces available for the experiments:
\begin{enumerate*}
	\item [\textbf{op0}:] First operator for the nodes in the given country.%First SIM card for the node.
	\item [\textbf{op1}:] First operator for the nodes in the given country.%Second SIM card for the node.
	\item [\textbf{op2}:] First operator for the nodes in the given country.%Third SIM card for the node.
	\item [\textbf{eth0}:] Ethernet (wired) network connection, when available.
\end{enumerate*}
%For a given country, the platform guarantees that a given SIM card index does always correspond to the same operator, unless a change of operators is performed at some future point.
For a given country, the platform guarantees that a given op$_i$ does always correspond to the same operator.

Under some circumstances, the mobile devices used in the \monroe{} nodes may loose connectivity, reset themselves or undergo any other process that makes them temporarily unavailable for the experiments.
To identify and tackle with these situations, experimenters are encouraged to build ``robust'' experiments subscribing to the corresponding metadata streams.

If the experimenter writes their own code:
\begin{enumerate*}
	\item Subscribe to the metadata broadcast.
	\item Wait for a \identifier{MODEM.*.UPDATE} message for the modem(s)/operators of interest.
	\item Once this information is obtained, use the desired interface and store the results with the corresponding ICCID or operator name.
	\item If the interface should disappear (\identifier{ENODEV} error, ``no such device''), start over at 2.
\end{enumerate*}
	
When an external tool that does not handle \identifier{ENODEV} (e.g., ``fping''), replace step 4 by:
\begin{itemize*}
	\item Monitor the metadata for a \identifier{MODEM.*.CONNECTIVITY} message indicating that connectivity was lost, or monitor the interface list to check if the device disappears. Upon either event, start over at 2.
\end{itemize*}
	
Experimenters should take notice that an interface may not only go down, but it may actually disappear from the list of available interfaces (if the modem has to be restarted, or anything similar happens in the USB stack).
Even if it reappears soon after, any existing network connections on the old interface will fail with \identifier{ENODEV}. 
	
It is also possible to skip steps 1 and 2 when reconnecting to an interface after a failure, as the interface name corresponding to the desired operator is already known.
It is still necessary to keep retrying to connect to the interface, until it comes up.


\subsection{Metadata at run-time}
\label{subsec:metadataRunTime}

\monroe{} nodes retrieve constantly some metadata information concerning their own state and the network conditions.
This information is continuously uploaded to the \monroe{} servers and stored in a database.
One of the main goals of the \monroe{} project is to make all that information freely accessible.
Therefore, experimenters may perform an off-line correlations of events in their experiment with the information in the \monroe{} database.

\monroe{} experimenters can also access all the metadata information at run-time from their experiments to achieve easy correlation of events or modify the behavior of the experiment during its execution. For example:
\begin{itemize*}
	\item Experiments that depend on external factors (location):
	\begin{itemize*}
		\item Round trip time vs. location.
		\item Proactive HTTP caching according to location.
		\item Round trip time vs. base station.
		\item Round trip time vs. signal strength.
		\item Route selection according to current conditions.
	\end{itemize*}
	\item Experiment validation:
	\begin{itemize*}
		\item Verify that node temperature is/was within limits.
		\item Verify that system load is/was below threshold.
	\end{itemize*}	
\end{itemize*}

The metadata is broadcast locally using ZeroMQ.
The following excerpt in Python shows how an application can subscribe to the metadata stream:
{\VerbatimFont
\begin{verbatim}
import zmq

context = zmq.Context()
socket = context.socket(zmq.SUB)
socket.connect ("tcp://172.17.0.1:5556")

# An empty string subscribes to everything:
topicfilter = ''   # E.g., use 'MONROE.META.DEVICE.GPS' for GPS-only metadata
socket.setsockopt(zmq.SUBSCRIBE, topicfilter)

while True:
  string = socket.recv()
  print string
\end{verbatim}
}


\subsubsection{Example: Correlate experiment results with metadata at run-time}

The following example shows how to create an application that executes a ping to an external machine and saves the results alongside the node location:

\begin{itemize*}
	\item Pipe the ping command through a ``ping formatter.''
	\item The ``ping'' formatter subscribes to a zmq socket and topic:
	\begin{itemize*}
		\item Socket : 'tcp://172.17.0.1:5556'
		\item Topic : 'MONROE.META.DEVICE.GPS' 
	\end{itemize*}
	\item Cache the GPS position received. 
	\item Wait for output from the ping command (stdin).
	\item Store experiment information including the GPS position:
	\begin{itemize*}
		\item Use the ``library'' \identifier{monroe\_exporter} (python only).
		\item Call the \identifier{monroe\_exporter} script via the command line.
	\end{itemize*}
\end{itemize*}

Below is the corresponding source code:
{\VerbatimFont
\begin{verbatim}
socket.connect('tcp://localhost:5557')
socket.setsockopt(zmq.SUBSCRIBE, 'MONROE.META.DEVICE.GPS')
LAST_GPS_FIX = None

monroe_exporter.initalize('MONROE.EXP.PING', 1, 5.0)

'''fork and wait for for gps messages'''
while True:
  (topic, msgdata) = socket.recv_multipart()
  LAST_GPS_FIX = json.loads(msgdata)

'''main process waits for ping experiment output '''
while line:
  exp_result = r.match(line).groupdict()
  msg = {
    'InterfaceName': interface,
    'Bytes': int(exp_result['bytes']),
    'Host': exp_result['host'],
    'Rtt': float(exp_result['rtt']),
    'SequenceNumber': int(exp_result['seq']),
    'TimeStamp': float(exp_result['ts'])
  }
  if LAST_GPS_FIX != None:
    msg.update(
      {
        'GPSTimeStamp': LAST_GPS_FIX['TimeStamp'],
        'Latitude': LAST_GPS_FIX['Latitude'],
        'Longitude': LAST_GPS_FIX['Longitude'],
        'Altitude': LAST_GPS_FIX['Altitude'],
        'NumberofSatellites': LAST_GPS_FIX['NumberofSatellites']
          })

  monroe_exporter.save_output(msg)
  line = sys.stdin.readline()
\end{verbatim}
}
 
\subsubsection{Metadata information}

Currently, the collected metadata includes:
\begin{itemize*}
	\item Node GPS.
	\item Node sensors (CPU temp) and probes (load, memory usage).
	\item Modem events.
	\item Modem connectivity status.
	\item Continuous and scheduled internal experiments:
	\begin{itemize*}
		\item RTT (through ping).
		\item Bandwidth (through HTTP download).
	\end{itemize*}
\end{itemize*}

The following and some examples of the information received in the metadata stream:

\begin{itemize}
	\item RTT experiment:\\
	\texttt{\footnotesize\{"\textbf{DataId}": "MONROE.EXP.PING", "\textbf{Bytes}": 84, "\textbf{NodeId}": "54", \\
		"\textbf{SequenceNumber}": 301, "\textbf{DataVersion}": 1, "\textbf{Timestamp}": 1465805479.747943, \\
		"\textbf{Rtt}": 71.2, "\textbf{Host}": "8.8.8.8", "\textbf{Operator}": "Orange", \\
		"\textbf{Iccid}": "8934014251541036013", "\textbf{Guid}": \\ "sha256:a9f9fb2c04bba3782ef2624e118faa18f16b08c826155cae5e1ea7e1d88832b5.0.54.3791"\}}

	\item Sensors, where each message may contain information about a different set of measurements:\\
	\texttt{\footnotesize\{"\textbf{DataId}": "MONROE.META.NODE.SENSOR", "\textbf{softirq}": "205270", "\textbf{SequenceNumber}": 48581,\\
		"\textbf{DataVersion}": 1, "b": "1059270", "b": "4885494", "\textbf{guest}": "0", "\textbf{NodeId}": "54", \\
		"\textbf{idle}": "42657942", "\textbf{user}": "10480984", "\textbf{irq}": "0", "\textbf{steal}": "0", \\
		"\textbf{Timestamp}": 1465786966, "\textbf{nice}": "3063"\}}\\~\\
	\texttt{\footnotesize\{"\textbf{DataId}": "MONROE.META.NODE.SENSOR", "\textbf{SequenceNumber}": 48567, "\textbf{DataVersion}": 1,\\
		"\textbf{Timestamp}": 1465786961, "\textbf{percent}": "65.98", "\textbf{NodeId}": "54", "\textbf{current}": "302234", \\
		"\textbf{start}": "1465484726", "\textbf{total}": "5246545.72", "\textbf{id}": "39"\}}\\~\\
	\texttt{\footnotesize\{"\textbf{DataId}": "MONROE.META.NODE.SENSOR", "\textbf{SequenceNumber}": 48460, "\textbf{DataVersion}": 1,\\
		"\textbf{Timestamp}": 1465786926, "\textbf{apps}": "3632746496", "\textbf{NodeId}": "54", "\textbf{free}": "483119104", \\
		"\textbf{swap}": "0"\}}
	
	\item Modem events:\\
	\texttt{\footnotesize\{"\textbf{DataId}": "MONROE.META.DEVICE.MODEM", "\textbf{InterfaceName}": "usb2", "\textbf{CID}": 72209509,\\
		"\textbf{DeviceState}": 3, "\textbf{SequenceNumber}": 33548, "\textbf{DataVersion}": 1, "\textbf{Timestamp}": 1465803136, \\
		"\textbf{NWMCCMNC}": 21404, "\textbf{Band}": 3, "\textbf{RSSI}": -80, "\textbf{IPAddress}": "10.33.101.173", \\
		"\textbf{IMSIMCCMNC}": 21404, "\textbf{DeviceMode}": 5, "\textbf{NodeId}": "54", "\textbf{IMEI}": "864154023645179", \\
		"\textbf{RSRQ}": -8, "\textbf{RSRP}": -85, "\textbf{LAC}": 28014, "\textbf{Frequency}": 1800, \\
		"\textbf{InternalIPAddress}": "192.168.0.153", "\textbf{Operator}": "YOIGO", \\
		"\textbf{ICCID}": "8934041514050774002", "\textbf{IMSI}": "214040113950108"\}}
	
	\item Connectivity:\\
	\texttt{\footnotesize\{"\textbf{DataId}": "MONROE.META.CONNECTIVITY", "\textbf{InterfaceName}": "usb2", "\textbf{SequenceNumber}": 41412, \\
		"\textbf{DataVersion}": 1, "\textbf{Timestamp}": 1465805641, "\textbf{NodeId}": "54", "\textbf{MCCMNC}": 21404, \\
		"\textbf{Mode}": 6, "\textbf{RSSI}": -80, "\textbf{ICCID}": "8934041514050774002"\}}
	
	\item GPS:\\
	\texttt{\footnotesize\{"\textbf{DataId}": "MONROE.META.DEVICE.GPS", "\textbf{SequenceNumber}": 34164, "\textbf{DataVersion}": 1, \\
		"\textbf{Timestamp}": 1465805718, "\textbf{Altitude}": -1455.900024, "\textbf{NodeId}": "63", \\
		"\textbf{Longitude}": -3.777019, "\textbf{NMEA}":\\ "\$GPGGA,081518.0,4020.002011,N,00346.621107,W,1,02,500.0,-1455.9,M,53.0,M,,*5D\textbackslash{}r\textbackslash{}n",\\
		"\textbf{SatelliteCount}": 2, "\textbf{Latitude}": 40.333366\}}
	
\end{itemize}

\subsubsection{Metadata format}
Metadata and internal experiment results follow a JSON structure, as detailed in \url{https://secure.mlab.no/monroewiki/doku.php?id=dataformat}~:
\begin{itemize*}
	\item All Metadata messages have a topic according to Table~\ref{tab:metadataTopics} (\url{https://secure.mlab.no/monroewiki/doku.php?id=metadataformat}). Appendix~\ref{app:metadataFields} gives the complete description of the meaning of all the metadata fields.
	\item All metadata topics are prefixed with ``MONROE.META.''
	\item All internal experiments are prefixed with ``MONROE.EXP.''
	\item Experiments receive metadata messages only for topics to which they subscribe.
	\item An empty string ("") subscribes to all topics.
\end{itemize*}

\begin{table}[tp]
	\caption{Metadata topics.}\label{tab:metadataTopics}
	\begin{center}
	\begin{tabular}{ll}
		\toprule
		\textbf{TOPIC} & \textbf{DESCRIPTION} \\
		\midrule
		*.CONNECTIVITY.iccid & Interface, connection events, IP \\
		*.DEVICE.MODEM.iccid.UPDATE	& \\
		*.DEVICE.MODEM.iccid.MODE & \\
		*.DEVICE.MODEM.iccid.SIGNAL	& \\
		*.DEVICE.MODEM.iccid.LTEBAND	& \\
		*.DEVICE.MODEM.iccid.ISPNAME	& \\
		*.DEVICE.MODEM.iccid.IPADDR	& \\
		*.DEVICE.MODEM.iccid.LOCCHANGE	& \\
		*.DEVICE.MODEM.iccid.NWMCCMNCCHANGE	& \\
		*.DEVICE.GPS	& \\
		*.NODE.SENSOR.sensor\_name & Temp sensor, running experiments, quotas, \ldots \\
		*.NODE.EVENT & Power up events, etc, \ldots \\
		\bottomrule
	\end{tabular}
	\end{center}
\end{table}

%------------------------------------------------------------%

\section{Node status}
\label{subsec:nodeStatus}

The user can check the state of the nodes under the tab ``Resources.'' Figure~\ref{fig:Resources} shows an example of the information that is supplied.

\begin{figure}[tp]
	\centering
	\includegraphics[width=1\textwidth]{Resources.png}
	\caption{Status of the \monroe{} nodes.}
	\label{fig:Resources}
\end{figure}

%------------------------------------------------------------%

\section{List of known bugs and issues}

\begin{itemize}
	
	\item In general, Firefox does not render the date-time picker correctly. You will have to either enter the dates and times manually or use Chrome.
	\item Container deployment can take several minutes, particularly for nodes without an Ethernet management connection (e.g., mobile nodes in trains or buses). When scheduling an experiment, the user has to take into account the time needed for the deployment. The system will not automatically take care of this at this moment.
	\item Similarly, the button ``Check availability'' returns the earliest available slot. However, it does not account for the time needed to deploy the container. The user must manually account for that.
	\item Checking the option ``ASAP'' to schedule an experiment as soon as possible may fail due to lack of time to deploy the container. The system does add some slack in this case, but its length may need some adjustment according to the type of nodes and MBB characteristics.
\end{itemize}

%------------------------------------------------------------%

\begin{appendices}
\section{List of packages installed in monroe/base}
\label{app:installedPackages}

{\scriptsize
\begin{longtable}{p{3.25cm}@{\hspace{0.25cm}}p{4cm}@{\hspace{0.25cm}}l@{\hspace{0.25cm}}p{7cm}}
	\caption{List of packages installed in \identifier{monroe/base}.}\label{tab:installedPackages}\\
	\toprule
	\textbf{Name} & \textbf{Version} & \textbf{Architecture} & \textbf{Description} \\	\midrule
	\endfirsthead
	\caption{List of packages installed in \identifier{monroe/base}. (Continued)}\\
	\toprule
	\textbf{Name} & \textbf{Version} & \textbf{Architecture} & \textbf{Description} \\	\midrule
	\endhead
%
acl	&	2.2.52-2	&	amd64	&	Access control list utilities	\\
adduser	&	3.113+nmu3	&	all	&	add and remove users and groups	\\
apache2	&	2.4.10-10+deb8u4	&	amd64	&	Apache HTTP Server	\\
apache2-bin	&	2.4.10-10+deb8u4	&	amd64	&	Apache HTTP Server (modules and other binary files)	\\
apache2-data	&	2.4.10-10+deb8u4	&	all	&	Apache HTTP Server (common files)	\\
apache2-utils	&	2.4.10-10+deb8u4	&	amd64	&	Apache HTTP Server (utility programs for web servers)	\\
apt	&	1.0.9.8.3	&	amd64	&	commandline package manager	\\
base-files	&	8+deb8u4	&	amd64	&	Debian base system miscellaneous files	\\
base-passwd	&	3.5.37	&	amd64	&	Debian base system master password and group files	\\
bash	&	4.3-11+b1	&	amd64	&	GNU Bourne Again SHell	\\
bind9-host	&	1:9.9.5.dfsg-9+deb8u6	&	amd64	&	Version of 'host' bundled with BIND 9.X	\\
blt	&	2.5.3+dfsg-1	&	amd64	&	graphics extension library for Tcl/Tk - run-time	\\
bsdutils	&	1:2.25.2-6	&	amd64	&	basic utilities from 4.4BSD-Lite	\\
ca-certificates	&	20141019+deb8u1	&	all	&	Common CA certificates	\\
ca-certificates-java	&	20140324	&	all	&	Common CA certificates (JKS keystore)	\\
coreutils	&	8.23-4	&	amd64	&	GNU core utilities	\\
curl	&	7.38.0-4+deb8u3	&	amd64	&	command line tool for transferring data with URL syntax	\\
d-itg	&	2.8.1-r1023-3	&	amd64	&	Distributed Internet Traffic Generator	\\
dash	&	0.5.7-4+b1	&	amd64	&	POSIX-compliant shell	\\
dbus	&	1.8.20-0+deb8u1	&	amd64	&	simple interprocess messaging system (daemon and utilities)	\\
debconf	&	1.5.56	&	all	&	Debian configuration management system	\\
debconf-i18n	&	1.5.56	&	all	&	full internationalization support for debconf	\\
debian-archive-keyring	&	2014.3	&	all	&	GnuPG archive keys of the Debian archive	\\
debianutils	&	4.4+b1	&	amd64	&	Miscellaneous utilities specific to Debian	\\
default-jre-headless	&	2:1.7-52	&	amd64	&	Standard Java or Java compatible Runtime (headless)	\\
diffutils	&	1:3.3-1+b1	&	amd64	&	File comparison utilities	\\
dmsetup	&	2:1.02.90-2.2	&	amd64	&	Linux Kernel Device Mapper userspace library	\\
dnsutils	&	1:9.9.5.dfsg-9+deb8u6	&	amd64	&	Clients provided with BIND	\\
dpkg	&	1.17.26	&	amd64	&	Debian package management system	\\
e2fslibs:amd64	&	1.42.12-1.1	&	amd64	&	ext2/ext3/ext4 file system libraries	\\
e2fsprogs	&	1.42.12-1.1	&	amd64	&	ext2/ext3/ext4 file system utilities	\\
echoping	&	6.0.2-8	&	amd64	&	Small test tool for TCP servers	\\
file	&	1:5.22+15-2+deb8u1	&	amd64	&	Determines file type using "magic" numbers	\\
findutils	&	4.4.2-9+b1	&	amd64	&	utilities for finding files--find, xargs	\\
flent	&	0.14.0-1	&	all	&	The FLExible Network Tester	\\
fontconfig	&	2.11.0-6.3	&	amd64	&	generic font configuration library - support binaries	\\
fontconfig-config	&	2.11.0-6.3	&	all	&	generic font configuration library - configuration	\\
fonts-dejavu-core	&	2.34-1	&	all	&	Vera font family derivate with additional characters	\\
fonts-lyx	&	2.1.2-2	&	all	&	TrueType versions of some TeX fonts used by LyX	\\
fping	&	3.10-2	&	amd64	&	sends ICMP ECHO\_REQUEST packets to network hosts	\\
gcc-4.8-base:amd64	&	4.8.4-1	&	amd64	&	GCC, the GNU Compiler Collection (base package)	\\
gcc-4.9-base:amd64	&	4.9.2-10	&	amd64	&	GCC, the GNU Compiler Collection (base package)	\\
geoip-database	&	20150317-1	&	all	&	IP lookup command line tools that use the GeoIP library (country database)	\\
gnupg	&	1.4.18-7+deb8u1	&	amd64	&	GNU privacy guard - a free PGP replacement	\\
gpgv	&	1.4.18-7+deb8u1	&	amd64	&	GNU privacy guard - signature verification tool	\\
gpsd	&	3.11-3	&	amd64	&	Global Positioning System - daemon	\\
gpslogger-oml2	&	2.11.0-mytestbed2	&	amd64	&	Record and store GPS measurements using OML	\\
grep	&	2.20-4.1	&	amd64	&	GNU grep, egrep and fgrep	\\
gstreamer1.0-plugins-base:amd64	&	1.4.4-2	&	amd64	&	GStreamer plugins from the "base" set	\\
gzip	&	1.6-4	&	amd64	&	GNU compression utilities	\\
hicolor-icon-theme	&	0.13-1	&	all	&	default fallback theme for FreeDesktop.org icon themes	\\
hostname	&	3.15	&	amd64	&	utility to set/show the host name or domain name	\\
httperf-oml2	&	2.11.0-mytestbed2	&	amd64	&	HTTP server performance tester, with OML support	\\
httping	&	1.5.8-1	&	amd64	&	ping-like program for http-requests	\\
inetutils-ping	&	2:1.9.2.39.3a460-3	&	amd64	&	ICMP echo tool	\\
init	&	1.22	&	amd64	&	System-V-like init utilities - metapackage	\\
init-system-helpers	&	1.22	&	all	&	helper tools for all init systems	\\
initscripts	&	2.88dsf-59	&	amd64	&	scripts for initializing and shutting down the system	\\
insserv	&	1.14.0-5	&	amd64	&	boot sequence organizer using LSB init.d script dependency information	\\
iperf	&	2.0.5+dfsg1-2	&	amd64	&	Internet Protocol bandwidth measuring tool	\\
iperf-oml2	&	2.11.0-mytestbed2	&	amd64	&	Internet Protocol bandwidth measuring tool, with OML support	\\
iperf3	&	3.0.7-1	&	amd64	&	Internet Protocol bandwidth measuring tool	\\
iproute2	&	3.16.0-2	&	amd64	&	networking and traffic control tools	\\
iptables	&	1.4.21-2+b1	&	amd64	&	administration tools for packet filtering and NAT	\\
iso-codes	&	3.57-1	&	all	&	ISO language, territory, currency, script codes and their translations	\\
java-common	&	0.52	&	all	&	Base of all Java packages	\\
javascript-common	&	11	&	all	&	Base support for JavaScript library packages	\\
krb5-locales	&	1.12.1+dfsg-19+deb8u2	&	all	&	Internationalization support for MIT Kerberos	\\
libacl1:amd64	&	2.2.52-2	&	amd64	&	Access control list shared library	\\
libalgorithm-c3-perl	&	0.09-1	&	all	&	Perl module for merging hierarchies using the C3 algorithm	\\
libapr1:amd64	&	1.5.1-3	&	amd64	&	Apache Portable Runtime Library	\\
libaprutil1:amd64	&	1.5.4-1	&	amd64	&	Apache Portable Runtime Utility Library	\\
libaprutil1-dbd-sqlite3:amd64	&	1.5.4-1	&	amd64	&	Apache Portable Runtime Utility Library - SQLite3 Driver	\\
libaprutil1-ldap:amd64	&	1.5.4-1	&	amd64	&	Apache Portable Runtime Utility Library - LDAP Driver	\\
libapt-pkg4.12:amd64	&	1.0.9.8.3	&	amd64	&	package management runtime library	\\
libarchive-extract-perl	&	0.72-1	&	all	&	generic archive extracting module	\\
libasound2:amd64	&	1.0.28-1	&	amd64	&	shared library for ALSA applications	\\
libasound2-data	&	1.0.28-1	&	all	&	Configuration files and profiles for ALSA drivers	\\
libasyncns0:amd64	&	0.8-5	&	amd64	&	Asynchronous name service query library	\\
libatk1.0-0:amd64	&	2.14.0-1	&	amd64	&	ATK accessibility toolkit	\\
libatk1.0-data	&	2.14.0-1	&	all	&	Common files for the ATK accessibility toolkit	\\
libattr1:amd64	&	1:2.4.47-2	&	amd64	&	Extended attribute shared library	\\
libaudio2:amd64	&	1.9.4-3	&	amd64	&	Network Audio System - shared libraries	\\
libaudit-common	&	1:2.4-1	&	all	&	Dynamic library for security auditing - common files	\\
libaudit1:amd64	&	1:2.4-1+b1	&	amd64	&	Dynamic library for security auditing	\\
libauthen-sasl-perl	&	2.1600-1	&	all	&	Authen::SASL - SASL Authentication framework	\\
libavahi-client3:amd64	&	0.6.31-5	&	amd64	&	Avahi client library	\\
libavahi-common-data:amd64	&	0.6.31-5	&	amd64	&	Avahi common data files	\\
libavahi-common3:amd64	&	0.6.31-5	&	amd64	&	Avahi common library	\\
libbind9-90	&	1:9.9.5.dfsg-9+deb8u6	&	amd64	&	BIND9 Shared Library used by BIND	\\
libblas-common	&	1.2.20110419-10	&	amd64	&	Dependency package for all BLAS implementations	\\
libblas3	&	1.2.20110419-10	&	amd64	&	Basic Linear Algebra Reference implementations, shared library	\\
libblkid1:amd64	&	2.25.2-6	&	amd64	&	block device id library	\\
libbluetooth3:amd64	&	5.23-2+b1	&	amd64	&	Library to use the BlueZ Linux Bluetooth stack	\\
libbsd0:amd64	&	0.7.0-2	&	amd64	&	utility functions from BSD systems - shared library	\\
libbz2-1.0:amd64	&	1.0.6-7+b3	&	amd64	&	high-quality block-sorting file compressor library - runtime	\\
libc-bin	&	2.19-18+deb8u4	&	amd64	&	GNU C Library: Binaries	\\
libc6:amd64	&	2.19-18+deb8u4	&	amd64	&	GNU C Library: Shared libraries	\\
libcairo2:amd64	&	1.14.0-2.1+deb8u1	&	amd64	&	Cairo 2D vector graphics library	\\
libcap-ng0:amd64	&	0.7.4-2	&	amd64	&	An alternate POSIX capabilities library	\\
libcap2:amd64	&	1:2.24-8	&	amd64	&	POSIX 1003.1e capabilities (library)	\\
libcap2-bin	&	1:2.24-8	&	amd64	&	POSIX 1003.1e capabilities (utilities)	\\
libcdparanoia0:amd64	&	3.10.2+debian-11	&	amd64	&	audio extraction tool for sampling CDs (library)	\\
libcgi-fast-perl	&	1:2.04-1	&	all	&	CGI subclass for work with FCGI	\\
libcgi-pm-perl	&	4.09-1	&	all	&	module for Common Gateway Interface applications	\\
libclass-c3-perl	&	0.26-1	&	all	&	pragma for using the C3 method resolution order	\\
libclass-c3-xs-perl	&	0.13-2+b1	&	amd64	&	Perl module to accelerate Class::C3	\\
libcomerr2:amd64	&	1.42.12-1.1	&	amd64	&	common error description library	\\
libconfig-grammar-perl	&	1.10-2	&	all	&	grammar-based user-friendly config parser	\\
libcpan-meta-perl	&	2.142690-1	&	all	&	Perl module to access CPAN distributions metadata	\\
libcryptsetup4:amd64	&	2:1.6.6-5	&	amd64	&	disk encryption support - shared library	\\
libcups2:amd64	&	1.7.5-11+deb8u1	&	amd64	&	Common UNIX Printing System(tm) - Core library	\\
libcurl3:amd64	&	7.38.0-4+deb8u3	&	amd64	&	easy-to-use client-side URL transfer library (OpenSSL flavour)	\\
libdata-optlist-perl	&	0.109-1	&	all	&	module to parse and validate simple name/value option pairs	\\
libdata-section-perl	&	0.200006-1	&	all	&	module to read chunks of data from a module's DATA section	\\
libdatrie1:amd64	&	0.2.8-1	&	amd64	&	Double-array trie library	\\
libdb5.3:amd64	&	5.3.28-9	&	amd64	&	Berkeley v5.3 Database Libraries [runtime]	\\
libdbi1:amd64	&	0.9.0-4	&	amd64	&	DB Independent Abstraction Layer for C -- shared library	\\
libdbus-1-3:amd64	&	1.8.20-0+deb8u1	&	amd64	&	simple interprocess messaging system (library)	\\
libdebconfclient0:amd64	&	0.192	&	amd64	&	Debian Configuration Management System (C-implementation library)	\\
libdevmapper1.02.1:amd64	&	2:1.02.90-2.2	&	amd64	&	Linux Kernel Device Mapper userspace library	\\
libdigest-hmac-perl	&	1.03+dfsg-1	&	all	&	module for creating standard message integrity checks	\\
libdns100	&	1:9.9.5.dfsg-9+deb8u6	&	amd64	&	DNS Shared Library used by BIND	\\
libdrm-intel1:amd64	&	2.4.58-2	&	amd64	&	Userspace interface to intel-specific kernel DRM services -- runtime	\\
libdrm-nouveau2:amd64	&	2.4.58-2	&	amd64	&	Userspace interface to nouveau-specific kernel DRM services -- runtime	\\
libdrm-radeon1:amd64	&	2.4.58-2	&	amd64	&	Userspace interface to radeon-specific kernel DRM services -- runtime	\\
libdrm2:amd64	&	2.4.58-2	&	amd64	&	Userspace interface to kernel DRM services -- runtime	\\
libedit2:amd64	&	3.1-20140620-2	&	amd64	&	BSD editline and history libraries	\\
libelf1:amd64	&	0.159-4.2	&	amd64	&	library to read and write ELF files	\\
libencode-locale-perl	&	1.03-1	&	all	&	utility to determine the locale encoding	\\
libexpat1:amd64	&	2.1.0-6+deb8u1	&	amd64	&	XML parsing C library - runtime library	\\
libfcgi-perl	&	0.77-1+b1	&	amd64	&	helper module for FastCGI	\\
libffi6:amd64	&	3.1-2+b2	&	amd64	&	Foreign Function Interface library runtime	\\
libfile-listing-perl	&	6.04-1	&	all	&	module to parse directory listings	\\
libflac8:amd64	&	1.3.0-3	&	amd64	&	Free Lossless Audio Codec - runtime C library	\\
libfont-afm-perl	&	1.20-1	&	all	&	Font::AFM - Interface to Adobe Font Metrics files	\\
libfontconfig1:amd64	&	2.11.0-6.3	&	amd64	&	generic font configuration library - runtime	\\
libfreetype6:amd64	&	2.5.2-3+deb8u1	&	amd64	&	FreeType 2 font engine, shared library files	\\
libgcc1:amd64	&	1:4.9.2-10	&	amd64	&	GCC support library	\\
libgcrypt20:amd64	&	1.6.3-2+deb8u1	&	amd64	&	LGPL Crypto library - runtime library	\\
libgdbm3:amd64	&	1.8.3-13.1	&	amd64	&	GNU dbm database routines (runtime version)	\\
libgdk-pixbuf2.0-0:amd64	&	2.31.1-2+deb8u4	&	amd64	&	GDK Pixbuf library	\\
libgdk-pixbuf2.0-common	&	2.31.1-2+deb8u4	&	all	&	GDK Pixbuf library - data files	\\
libgeoip1:amd64	&	1.6.2-4	&	amd64	&	non-DNS IP-to-country resolver library	\\
libgfortran3:amd64	&	4.9.2-10	&	amd64	&	Runtime library for GNU Fortran applications	\\
libgl1-mesa-dri:amd64	&	10.3.2-1+deb8u1	&	amd64	&	free implementation of the OpenGL API -- DRI modules	\\
libgl1-mesa-glx:amd64	&	10.3.2-1+deb8u1	&	amd64	&	free implementation of the OpenGL API -- GLX runtime	\\
libglade2-0:amd64	&	1:2.6.4-2	&	amd64	&	library to load .glade files at runtime	\\
libglapi-mesa:amd64	&	10.3.2-1+deb8u1	&	amd64	&	free implementation of the GL API -- shared library	\\
libglib2.0-0:amd64	&	2.42.1-1+b1	&	amd64	&	GLib library of C routines	\\
libglib2.0-data	&	2.42.1-1	&	all	&	Common files for GLib library	\\
libglu1-mesa:amd64	&	9.0.0-2	&	amd64	&	Mesa OpenGL utility library (GLU)	\\
libgmp10:amd64	&	2:6.0.0+dfsg-6	&	amd64	&	Multiprecision arithmetic library	\\
libgnutls-deb0-28:amd64	&	3.3.8-6+deb8u3	&	amd64	&	GNU TLS library - main runtime library	\\
libgpg-error0:amd64	&	1.17-3	&	amd64	&	library for common error values and messages in GnuPG components	\\
libgps21:amd64	&	3.11-3	&	amd64	&	Global Positioning System - library	\\
libgraphite2-3:amd64	&	1.3.6-1~deb8u1	&	amd64	&	Font rendering engine for Complex Scripts -- library	\\
libgssapi-krb5-2:amd64	&	1.12.1+dfsg-19+deb8u2	&	amd64	&	MIT Kerberos runtime libraries - krb5 GSS-API Mechanism	\\
libgstreamer-plugins-base1.0-0:amd64	&	1.4.4-2	&	amd64	&	GStreamer libraries from the "base" set	\\
libgstreamer1.0-0:amd64	&	1.4.4-2	&	amd64	&	Core GStreamer libraries and elements	\\
libgtk2.0-0:amd64	&	2.24.25-3+deb8u1	&	amd64	&	GTK+ graphical user interface library	\\
libgtk2.0-bin	&	2.24.25-3+deb8u1	&	amd64	&	programs for the GTK+ graphical user interface library	\\
libgtk2.0-common	&	2.24.25-3+deb8u1	&	all	&	common files for the GTK+ graphical user interface library	\\
libharfbuzz0b:amd64	&	0.9.35-2	&	amd64	&	OpenType text shaping engine (shared library)	\\
libhogweed2:amd64	&	2.7.1-5+deb8u1	&	amd64	&	low level cryptographic library (public-key cryptos)	\\
libhtml-form-perl	&	6.03-1	&	all	&	module that represents an HTML form element	\\
libhtml-format-perl	&	2.11-1	&	all	&	module for transforming HTML into various formats	\\
libhtml-parser-perl	&	3.71-1+b3	&	amd64	&	collection of modules that parse HTML text documents	\\
libhtml-tagset-perl	&	3.20-2	&	all	&	Data tables pertaining to HTML	\\
libhtml-tree-perl	&	5.03-1	&	all	&	Perl module to represent and create HTML syntax trees	\\
libhttp-cookies-perl	&	6.01-1	&	all	&	HTTP cookie jars	\\
libhttp-daemon-perl	&	6.01-1	&	all	&	simple http server class	\\
libhttp-date-perl	&	6.02-1	&	all	&	module of date conversion routines	\\
libhttp-message-perl	&	6.06-1	&	all	&	perl interface to HTTP style messages	\\
libhttp-negotiate-perl	&	6.00-2	&	all	&	implementation of content negotiation	\\
libice6:amd64	&	2:1.0.9-1+b1	&	amd64	&	X11 Inter-Client Exchange library	\\
libidn11:amd64	&	1.29-1+b2	&	amd64	&	GNU Libidn library, implementation of IETF IDN specifications	\\
libio-html-perl	&	1.001-1	&	all	&	open an HTML file with automatic charset detection	\\
libio-socket-inet6-perl	&	2.72-1	&	all	&	object interface for AF\_INET6 domain sockets	\\
libio-socket-ssl-perl	&	2.002-2+deb8u1	&	all	&	Perl module implementing object oriented interface to SSL sockets	\\
libiperf0	&	3.0.7-1	&	amd64	&	Internet Protocol bandwidth measuring tool (runtime files)	\\
libisc95	&	1:9.9.5.dfsg-9+deb8u6	&	amd64	&	ISC Shared Library used by BIND	\\
libisccc90	&	1:9.9.5.dfsg-9+deb8u6	&	amd64	&	Command Channel Library used by BIND	\\
libisccfg90	&	1:9.9.5.dfsg-9+deb8u6	&	amd64	&	Config File Handling Library used by BIND	\\
libjasper1:amd64	&	1.900.1-debian1-2.4+deb8u1	&	amd64	&	JasPer JPEG-2000 runtime library	\\
libjbig0:amd64	&	2.1-3.1	&	amd64	&	JBIGkit libraries	\\
libjpeg62-turbo:amd64	&	1:1.3.1-12	&	amd64	&	libjpeg-turbo JPEG runtime library	\\
libjs-cropper	&	1.2.2-1	&	all	&	JavaScript image cropper UI	\\
libjs-jquery	&	1.7.2+dfsg-3.2	&	all	&	JavaScript library for dynamic web applications	\\
libjs-jquery-ui	&	1.10.1+dfsg-1	&	all	&	JavaScript UI library for dynamic web applications	\\
libjs-prototype	&	1.7.1-3	&	all	&	JavaScript Framework for dynamic web applications	\\
libjs-scriptaculous	&	1.9.0-2	&	all	&	JavaScript library for dynamic web applications	\\
libjson-c2:amd64	&	0.11-4	&	amd64	&	JSON manipulation library - shared library	\\
libk5crypto3:amd64	&	1.12.1+dfsg-19+deb8u2	&	amd64	&	MIT Kerberos runtime libraries - Crypto Library	\\
libkeyutils1:amd64	&	1.5.9-5+b1	&	amd64	&	Linux Key Management Utilities (library)	\\
libkmod2:amd64	&	18-3	&	amd64	&	libkmod shared library	\\
libkrb5-3:amd64	&	1.12.1+dfsg-19+deb8u2	&	amd64	&	MIT Kerberos runtime libraries	\\
libkrb5support0:amd64	&	1.12.1+dfsg-19+deb8u2	&	amd64	&	MIT Kerberos runtime libraries - Support library	\\
liblapack3	&	3.5.0-4	&	amd64	&	Library of linear algebra routines 3 - shared version	\\
liblcms2-2:amd64	&	2.6-3+b3	&	amd64	&	Little CMS 2 color management library	\\
libldap-2.4-2:amd64	&	2.4.40+dfsg-1+deb8u2	&	amd64	&	OpenLDAP libraries	\\
liblinear1:amd64	&	1.8+dfsg-4	&	amd64	&	Library for Large Linear Classification	\\
libllvm3.5:amd64	&	1:3.5-10	&	amd64	&	Modular compiler and toolchain technologies, runtime library	\\
liblocale-gettext-perl	&	1.05-8+b1	&	amd64	&	module using libc functions for internationalization in Perl	\\
liblog-message-perl	&	0.8-1	&	all	&	powerful and flexible message logging mechanism	\\
liblog-message-simple-perl	&	0.10-2	&	all	&	simplified interface to Log::Message	\\
liblua5.1-0:amd64	&	5.1.5-7.1	&	amd64	&	Shared library for the Lua interpreter version 5.1	\\
liblua5.2-0:amd64	&	5.2.3-1.1	&	amd64	&	Shared library for the Lua interpreter version 5.2	\\
liblwp-mediatypes-perl	&	6.02-1	&	all	&	module to guess media type for a file or a URL	\\
liblwp-protocol-https-perl	&	6.06-2	&	all	&	HTTPS driver for LWP::UserAgent	\\
liblwres90	&	1:9.9.5.dfsg-9+deb8u6	&	amd64	&	Lightweight Resolver Library used by BIND	\\
liblzma5:amd64	&	5.1.1alpha+20120614-2+b3	&	amd64	&	XZ-format compression library	\\
libmagic1:amd64	&	1:5.22+15-2+deb8u1	&	amd64	&	File type determination library using "magic" numbers	\\
libmailtools-perl	&	2.13-1	&	all	&	Manipulate email in perl programs	\\
libmng1:amd64	&	1.0.10+dfsg-3.1+b3	&	amd64	&	Multiple-image Network Graphics library	\\
libmodule-build-perl	&	0.421000-2	&	all	&	framework for building and installing Perl modules	\\
libmodule-pluggable-perl	&	5.1-1	&	all	&	module for giving  modules the ability to have plugins	\\
libmodule-signature-perl	&	0.73-1+deb8u2	&	all	&	module to manipulate CPAN SIGNATURE files	\\
libmount1:amd64	&	2.25.2-6	&	amd64	&	device mounting library	\\
libmro-compat-perl	&	0.12-1	&	all	&	mro::* interface compatibility for Perls < 5.9.5	\\
libmysqlclient18:amd64	&	5.5.47-0+deb8u1	&	amd64	&	MySQL database client library	\\
libncurses5:amd64	&	5.9+20140913-1+b1	&	amd64	&	shared libraries for terminal handling	\\
libncursesw5:amd64	&	5.9+20140913-1+b1	&	amd64	&	shared libraries for terminal handling (wide character support)	\\
libnet-http-perl	&	6.07-1	&	all	&	module providing low-level HTTP connection client	\\
libnet-smtp-ssl-perl	&	1.01-3	&	all	&	Perl module providing SSL support to Net::SMTP	\\
libnet-ssleay-perl	&	1.65-1+b1	&	amd64	&	Perl module for Secure Sockets Layer (SSL)	\\
libnettle4:amd64	&	2.7.1-5+deb8u1	&	amd64	&	low level cryptographic library (symmetric and one-way cryptos)	\\
libnfnetlink0:amd64	&	1.0.1-3	&	amd64	&	Netfilter netlink library	\\
libnspr4:amd64	&	2:4.10.7-1+deb8u1	&	amd64	&	NetScape Portable Runtime Library	\\
libnss3:amd64	&	2:3.17.2-1.1+deb8u2	&	amd64	&	Network Security Service libraries	\\
libocomm	&	2.11.1~rc-mytestbed1	&	amd64	&	OComm:  O? Communications Library (metapackage)	\\
libocomm-dev	&	2.11.1~rc-mytestbed1	&	amd64	&	OML measurement library headers	\\
libocomm1	&	2.11.1~rc-mytestbed1	&	amd64	&	OComm:  O? Communications Library	\\
libogg0:amd64	&	1.3.2-1	&	amd64	&	Ogg bitstream library	\\
liboml2	&	2.11.1~rc-mytestbed1	&	amd64	&	OML: The O? Measurement Library (metapackage)	\\
liboml2-9	&	2.11.1~rc-mytestbed1	&	amd64	&	OML: The O? Measurement Library	\\
liboml2-dev	&	2.11.1~rc-mytestbed1	&	amd64	&	OML measurement library headers	\\
liborc-0.4-0:amd64	&	1:0.4.22-1	&	amd64	&	Library of Optimized Inner Loops Runtime Compiler	\\
libp11-kit0:amd64	&	0.20.7-1	&	amd64	&	Library for loading and coordinating access to PKCS\#11 modules - runtime	\\
libpackage-constants-perl	&	0.04-1	&	all	&	List constants defined in a package	\\
libpam-modules:amd64	&	1.1.8-3.1+deb8u1+b1	&	amd64	&	Pluggable Authentication Modules for PAM	\\
libpam-modules-bin	&	1.1.8-3.1+deb8u1+b1	&	amd64	&	Pluggable Authentication Modules for PAM - helper binaries	\\
libpam-runtime	&	1.1.8-3.1+deb8u1	&	all	&	Runtime support for the PAM library	\\
libpam0g:amd64	&	1.1.8-3.1+deb8u1+b1	&	amd64	&	Pluggable Authentication Modules library	\\
libpango-1.0-0:amd64	&	1.36.8-3	&	amd64	&	Layout and rendering of internationalized text	\\
libpangocairo-1.0-0:amd64	&	1.36.8-3	&	amd64	&	Layout and rendering of internationalized text	\\
libpangoft2-1.0-0:amd64	&	1.36.8-3	&	amd64	&	Layout and rendering of internationalized text	\\
libparams-util-perl	&	1.07-2+b1	&	amd64	&	Perl extension for simple stand-alone param checking functions	\\
libpcap0.8:amd64	&	1.6.2-2	&	amd64	&	system interface for user-level packet capture	\\
libpciaccess0:amd64	&	0.13.2-3+b1	&	amd64	&	Generic PCI access library for X	\\
libpcre3:amd64	&	2:8.35-3.3+deb8u4	&	amd64	&	Perl 5 Compatible Regular Expression Library - runtime files	\\
libpcsclite1:amd64	&	1.8.13-1	&	amd64	&	Middleware to access a smart card using PC/SC (library)	\\
libpgm-5.1-0	&	5.1.118-1~dfsg-1	&	amd64	&	OpenPGM shared library	\\
libpixman-1-0:amd64	&	0.32.6-3	&	amd64	&	pixel-manipulation library for X and cairo	\\
libpng12-0:amd64	&	1.2.50-2+deb8u2	&	amd64	&	PNG library - runtime	\\
libpod-latex-perl	&	0.61-1	&	all	&	module to convert Pod data to formatted LaTeX	\\
libpod-readme-perl	&	0.11-1	&	all	&	Perl module to convert POD to README file	\\
libpopt0:amd64	&	1.16-10	&	amd64	&	lib for parsing cmdline parameters	\\
libpq5:amd64	&	9.4.6-0+deb8u1	&	amd64	&	PostgreSQL C client library	\\
libprocps3:amd64	&	2:3.3.9-9	&	amd64	&	library for accessing process information from /proc	\\
libpulse0:amd64	&	5.0-13	&	amd64	&	PulseAudio client libraries	\\
libpython-stdlib:amd64	&	2.7.9-1	&	amd64	&	interactive high-level object-oriented language (default python version)	\\
libpython2.7:amd64	&	2.7.9-2	&	amd64	&	Shared Python runtime library (version 2.7)	\\
libpython2.7-minimal:amd64	&	2.7.9-2	&	amd64	&	Minimal subset of the Python language (version 2.7)	\\
libpython2.7-stdlib:amd64	&	2.7.9-2	&	amd64	&	Interactive high-level object-oriented language (standard library, version 2.7)	\\
libqt4-dbus:amd64	&	4:4.8.6+git64-g5dc8b2b+dfsg-3+deb8u1	&	amd64	&	Qt 4 D-Bus module	\\
libqt4-declarative:amd64	&	4:4.8.6+git64-g5dc8b2b+dfsg-3+deb8u1	&	amd64	&	Qt 4 Declarative module	\\
libqt4-designer:amd64	&	4:4.8.6+git64-g5dc8b2b+dfsg-3+deb8u1	&	amd64	&	Qt 4 designer module	\\
libqt4-help:amd64	&	4:4.8.6+git64-g5dc8b2b+dfsg-3+deb8u1	&	amd64	&	Qt 4 help module	\\
libqt4-network:amd64	&	4:4.8.6+git64-g5dc8b2b+dfsg-3+deb8u1	&	amd64	&	Qt 4 network module	\\
libqt4-opengl:amd64	&	4:4.8.6+git64-g5dc8b2b+dfsg-3+deb8u1	&	amd64	&	Qt 4 OpenGL module	\\
libqt4-script:amd64	&	4:4.8.6+git64-g5dc8b2b+dfsg-3+deb8u1	&	amd64	&	Qt 4 script module	\\
libqt4-scripttools:amd64	&	4:4.8.6+git64-g5dc8b2b+dfsg-3+deb8u1	&	amd64	&	Qt 4 script tools module	\\
libqt4-sql:amd64	&	4:4.8.6+git64-g5dc8b2b+dfsg-3+deb8u1	&	amd64	&	Qt 4 SQL module	\\
libqt4-sql-mysql:amd64	&	4:4.8.6+git64-g5dc8b2b+dfsg-3+deb8u1	&	amd64	&	Qt 4 MySQL database driver	\\
libqt4-svg:amd64	&	4:4.8.6+git64-g5dc8b2b+dfsg-3+deb8u1	&	amd64	&	Qt 4 SVG module	\\
libqt4-test:amd64	&	4:4.8.6+git64-g5dc8b2b+dfsg-3+deb8u1	&	amd64	&	Qt 4 test module	\\
libqt4-xml:amd64	&	4:4.8.6+git64-g5dc8b2b+dfsg-3+deb8u1	&	amd64	&	Qt 4 XML module	\\
libqt4-xmlpatterns:amd64	&	4:4.8.6+git64-g5dc8b2b+dfsg-3+deb8u1	&	amd64	&	Qt 4 XML patterns module	\\
libqtassistantclient4:amd64	&	4.6.3-6	&	amd64	&	Qt Assistant client library (runtime)	\\
libqtcore4:amd64	&	4:4.8.6+git64-g5dc8b2b+dfsg-3+deb8u1	&	amd64	&	Qt 4 core module	\\
libqtdbus4:amd64	&	4:4.8.6+git64-g5dc8b2b+dfsg-3+deb8u1	&	amd64	&	Qt 4 D-Bus module library	\\
libqtgui4:amd64	&	4:4.8.6+git64-g5dc8b2b+dfsg-3+deb8u1	&	amd64	&	Qt 4 GUI module	\\
libqtwebkit4:amd64	&	2.3.4.dfsg-3	&	amd64	&	Web content engine library for Qt	\\
libquadmath0:amd64	&	4.9.2-10	&	amd64	&	GCC Quad-Precision Math Library	\\
libreadline6:amd64	&	6.3-8+b3	&	amd64	&	GNU readline and history libraries, run-time libraries	\\
libregexp-common-perl	&	2013031301-1	&	all	&	module with common regular expressions	\\
librrd4	&	1.4.8-1.2	&	amd64	&	time-series data storage and display system (runtime library)	\\
librrds-perl	&	1.4.8-1.2	&	amd64	&	time-series data storage and display system (Perl interface, shared)	\\
librtmp1:amd64	&	2.4+20150115.gita107cef-1	&	amd64	&	toolkit for RTMP streams (shared library)	\\
libruby2.1:amd64	&	2.1.5-2+deb8u2	&	amd64	&	Libraries necessary to run Ruby 2.1	\\
libsasl2-2:amd64	&	2.1.26.dfsg1-13+deb8u1	&	amd64	&	Cyrus SASL - authentication abstraction library	\\
libsasl2-modules:amd64	&	2.1.26.dfsg1-13+deb8u1	&	amd64	&	Cyrus SASL - pluggable authentication modules	\\
libsasl2-modules-db:amd64	&	2.1.26.dfsg1-13+deb8u1	&	amd64	&	Cyrus SASL - pluggable authentication modules (DB)	\\
libsctp1:amd64	&	1.0.16+dfsg-2	&	amd64	&	user-space access to Linux Kernel SCTP - shared library	\\
libselinux1:amd64	&	2.3-2	&	amd64	&	SELinux runtime shared libraries	\\
libsemanage-common	&	2.3-1	&	all	&	Common files for SELinux policy management libraries	\\
libsemanage1:amd64	&	2.3-1+b1	&	amd64	&	SELinux policy management library	\\
libsepol1:amd64	&	2.3-2	&	amd64	&	SELinux library for manipulating binary security policies	\\
libsigar	&	1.6.5-1ppa1o	&	amd64	&	System Information Gatherer And Reporter	\\
libslang2:amd64	&	2.3.0-2	&	amd64	&	S-Lang programming library - runtime version	\\
libsm6:amd64	&	2:1.2.2-1+b1	&	amd64	&	X11 Session Management library	\\
libsmartcols1:amd64	&	2.25.2-6	&	amd64	&	smart column output alignment library	\\
libsndfile1:amd64	&	1.0.25-9.1+deb8u1	&	amd64	&	Library for reading/writing audio files	\\
libsnmp-session-perl	&	1.13-1.1	&	all	&	Perl support for accessing SNMP-aware devices	\\
libsocket6-perl	&	0.25-1+b1	&	amd64	&	Perl extensions for IPv6	\\
libsodium13:amd64	&	1.0.0-1	&	amd64	&	Network communication, cryptography and signaturing library	\\
libsoftware-license-perl	&	0.103010-3	&	all	&	module providing templated software licenses	\\
libsqlite3-0:amd64	&	3.8.7.1-1+deb8u1	&	amd64	&	SQLite 3 shared library	\\
libss2:amd64	&	1.42.12-1.1	&	amd64	&	command-line interface parsing library	\\
libssh2-1:amd64	&	1.4.3-4.1+deb8u1	&	amd64	&	SSH2 client-side library	\\
libssl1.0.0:amd64	&	1.0.1k-3+deb8u4	&	amd64	&	Secure Sockets Layer toolkit - shared libraries	\\
libstdc++6:amd64	&	4.9.2-10	&	amd64	&	GNU Standard C++ Library v3	\\
libsub-exporter-perl	&	0.986-1	&	all	&	sophisticated exporter for custom-built routines	\\
libsub-install-perl	&	0.928-1	&	all	&	module for installing subroutines into packages easily	\\
libsystemd0:amd64	&	215-17+deb8u4	&	amd64	&	systemd utility library	\\
libtasn1-6:amd64	&	4.2-3+deb8u1	&	amd64	&	Manage ASN.1 structures (runtime)	\\
libtcl8.6:amd64	&	8.6.2+dfsg-2	&	amd64	&	Tcl (the Tool Command Language) v8.6 - run-time library files	\\
libterm-ui-perl	&	0.42-1	&	all	&	Term::ReadLine UI made easy	\\
libtext-charwidth-perl	&	0.04-7+b3	&	amd64	&	get display widths of characters on the terminal	\\
libtext-iconv-perl	&	1.7-5+b2	&	amd64	&	converts between character sets in Perl	\\
libtext-soundex-perl	&	3.4-1+b2	&	amd64	&	implementation of the soundex algorithm	\\
libtext-template-perl	&	1.46-1	&	all	&	perl module to process text templates	\\
libtext-wrapi18n-perl	&	0.06-7	&	all	&	internationalized substitute of Text::Wrap	\\
libthai-data	&	0.1.21-1	&	all	&	Data files for Thai language support library	\\
libthai0:amd64	&	0.1.21-1	&	amd64	&	Thai language support library	\\
libtheora0:amd64	&	1.1.1+dfsg.1-6	&	amd64	&	Theora Video Compression Codec	\\
libtiff5:amd64	&	4.0.3-12.3+deb8u1	&	amd64	&	Tag Image File Format (TIFF) library	\\
libtimedate-perl	&	2.3000-2	&	all	&	collection of modules to manipulate date/time information	\\
libtinfo5:amd64	&	5.9+20140913-1+b1	&	amd64	&	shared low-level terminfo library for terminal handling	\\
libtk8.6:amd64	&	8.6.2-1	&	amd64	&	Tk toolkit for Tcl and X11 v8.6 - run-time files	\\
libtrace3	&	3.0.21-1	&	amd64	&	network trace processing library supporting many input formats	\\
libtxc-dxtn-s2tc0:amd64	&	0~git20131104-1.1	&	amd64	&	Texture compression library for Mesa	\\
libudev1:amd64	&	215-17+deb8u4	&	amd64	&	libudev shared library	\\
liburi-perl	&	1.64-1	&	all	&	module to manipulate and access URI strings	\\
libusb-0.1-4:amd64	&	2:0.1.12-25	&	amd64	&	userspace USB programming library	\\
libusb-1.0-0:amd64	&	2:1.0.19-1	&	amd64	&	userspace USB programming library	\\
libustr-1.0-1:amd64	&	1.0.4-3+b2	&	amd64	&	Micro string library: shared library	\\
libuuid1:amd64	&	2.25.2-6	&	amd64	&	Universally Unique ID library	\\
libvisual-0.4-0:amd64	&	0.4.0-6	&	amd64	&	Audio visualization framework	\\
libvisual-0.4-plugins:amd64	&	0.4.0.dfsg.1-7	&	amd64	&	Audio visualization framework plugins	\\
libvorbis0a:amd64	&	1.3.4-2	&	amd64	&	decoder library for Vorbis General Audio Compression Codec	\\
libvorbisenc2:amd64	&	1.3.4-2	&	amd64	&	encoder library for Vorbis General Audio Compression Codec	\\
libwandio1	&	3.0.21-1	&	amd64	&	multi-threaded file compression and decompression library	\\
libwebp5:amd64	&	0.4.1-1.2+b2	&	amd64	&	Lossy compression of digital photographic images.	\\
libwebpdemux1:amd64	&	0.4.1-1.2+b2	&	amd64	&	Lossy compression of digital photographic images.	\\
libwebpmux1:amd64	&	0.4.1-1.2+b2	&	amd64	&	Lossy compression of digital photographic images.	\\
libwrap0:amd64	&	7.6.q-25	&	amd64	&	Wietse Venema's TCP wrappers library	\\
libwww-perl	&	6.08-1	&	all	&	simple and consistent interface to the world-wide web	\\
libwww-robotrules-perl	&	6.01-1	&	all	&	database of robots.txt-derived permissions	\\
libx11-6:amd64	&	2:1.6.2-3	&	amd64	&	X11 client-side library	\\
libx11-data	&	2:1.6.2-3	&	all	&	X11 client-side library	\\
libx11-xcb1:amd64	&	2:1.6.2-3	&	amd64	&	Xlib/XCB interface library	\\
libxau6:amd64	&	1:1.0.8-1	&	amd64	&	X11 authorisation library	\\
libxcb-dri2-0:amd64	&	1.10-3+b1	&	amd64	&	X C Binding, dri2 extension	\\
libxcb-dri3-0:amd64	&	1.10-3+b1	&	amd64	&	X C Binding, dri3 extension	\\
libxcb-glx0:amd64	&	1.10-3+b1	&	amd64	&	X C Binding, glx extension	\\
libxcb-present0:amd64	&	1.10-3+b1	&	amd64	&	X C Binding, present extension	\\
libxcb-render0:amd64	&	1.10-3+b1	&	amd64	&	X C Binding, render extension	\\
libxcb-shm0:amd64	&	1.10-3+b1	&	amd64	&	X C Binding, shm extension	\\
libxcb-sync1:amd64	&	1.10-3+b1	&	amd64	&	X C Binding, sync extension	\\
libxcb1:amd64	&	1.10-3+b1	&	amd64	&	X C Binding	\\
libxcomposite1:amd64	&	1:0.4.4-1	&	amd64	&	X11 Composite extension library	\\
libxcursor1:amd64	&	1:1.1.14-1+b1	&	amd64	&	X cursor management library	\\
libxdamage1:amd64	&	1:1.1.4-2+b1	&	amd64	&	X11 damaged region extension library	\\
libxdmcp6:amd64	&	1:1.1.1-1+b1	&	amd64	&	X11 Display Manager Control Protocol library	\\
libxext6:amd64	&	2:1.3.3-1	&	amd64	&	X11 miscellaneous extension library	\\
libxfixes3:amd64	&	1:5.0.1-2+b2	&	amd64	&	X11 miscellaneous 'fixes' extension library	\\
libxft2:amd64	&	2.3.2-1	&	amd64	&	FreeType-based font drawing library for X	\\
libxi6:amd64	&	2:1.7.4-1+b2	&	amd64	&	X11 Input extension library	\\
libxinerama1:amd64	&	2:1.1.3-1+b1	&	amd64	&	X11 Xinerama extension library	\\
libxml2:amd64	&	2.9.1+dfsg1-5+deb8u1	&	amd64	&	GNOME XML library	\\
libxmuu1:amd64	&	2:1.1.2-1	&	amd64	&	X11 miscellaneous micro-utility library	\\
libxrandr2:amd64	&	2:1.4.2-1+b1	&	amd64	&	X11 RandR extension library	\\
libxrender1:amd64	&	1:0.9.8-1+b1	&	amd64	&	X Rendering Extension client library	\\
libxshmfence1:amd64	&	1.1-4	&	amd64	&	X shared memory fences - shared library	\\
libxslt1.1:amd64	&	1.1.28-2+b2	&	amd64	&	XSLT 1.0 processing library - runtime library	\\
libxss1:amd64	&	1:1.2.2-1	&	amd64	&	X11 Screen Saver extension library	\\
libxt6:amd64	&	1:1.1.4-1+b1	&	amd64	&	X11 toolkit intrinsics library	\\
libxtables10	&	1.4.21-2+b1	&	amd64	&	netfilter xtables library	\\
libxtst6:amd64	&	2:1.2.2-1+b1	&	amd64	&	X11 Testing -- Record extension library	\\
libxxf86vm1:amd64	&	1:1.1.3-1+b1	&	amd64	&	X11 XFree86 video mode extension library	\\
libyaml-0-2:amd64	&	0.1.6-3	&	amd64	&	Fast YAML 1.1 parser and emitter library	\\
libzmq3:amd64	&	4.0.5+dfsg-2+deb8u1	&	amd64	&	lightweight messaging kernel (shared library)	\\
lksctp-tools	&	1.0.16+dfsg-2	&	amd64	&	user-space access to Linux Kernel SCTP - commandline tools	\\
login	&	1:4.2-3+deb8u1	&	amd64	&	system login tools	\\
lsb-base	&	4.1+Debian13+nmu1	&	all	&	Linux Standard Base 4.1 init script functionality	\\
mawk	&	1.3.3-17	&	amd64	&	a pattern scanning and text processing language	\\
mgen	&	5.02+dfsg2-3	&	amd64	&	packet generator for IP network performance tests	\\
mime-support	&	3.58	&	all	&	MIME files 'mime.types' \& 'mailcap', and support programs	\\
mount	&	2.25.2-6	&	amd64	&	Tools for mounting and manipulating filesystems	\\
multiarch-support	&	2.19-18+deb8u4	&	amd64	&	Transitional package to ensure multiarch compatibility	\\
mysql-common	&	5.5.47-0+deb8u1	&	all	&	MySQL database common files, e.g. /etc/mysql/my.cnf	\\
ncurses-base	&	5.9+20140913-1	&	all	&	basic terminal type definitions	\\
ncurses-bin	&	5.9+20140913-1+b1	&	amd64	&	terminal-related programs and man pages	\\
ndiff	&	6.47-3	&	all	&	The Network Mapper - result compare utility	\\
net-tools	&	1.60-26+b1	&	amd64	&	NET-3 networking toolkit	\\
netbase	&	5.3	&	all	&	Basic TCP/IP networking system	\\
netperf	&	2.7.0-1	&	amd64	&	Network performance benchmark	\\
nmap	&	6.47-3+b1	&	amd64	&	The Network Mapper	\\
nmetrics-oml2	&	2.11.0-mytestbed2	&	amd64	&	Measure and record system information from libsigar using OML	\\
oml2	&	2.11.1~rc-mytestbed1	&	amd64	&	OML: The O? Measurement Library Suite (Metapackage)	\\
oml2-apps	&	2.11.0-mytestbed2	&	amd64	&	Standalone OML2 applications (metapackage)	\\
oml2-proxy-server	&	2.11.1~rc-mytestbed1	&	amd64	&	OML proxy server	\\
oml2-proxycon	&	2.11.1~rc-mytestbed1	&	amd64	&	OML proxy server control script	\\
oml2-server	&	2.11.1~rc-mytestbed1	&	amd64	&	OML measurement server	\\
openjdk-7-jre-headless:amd64	&	7u95-2.6.4-1~deb8u1	&	amd64	&	OpenJDK Java runtime, using Hotspot JIT (headless)	\\
openssh-client	&	1:6.7p1-5+deb8u2	&	amd64	&	secure shell (SSH) client, for secure access to remote machines	\\
openssl	&	1.0.1k-3+deb8u4	&	amd64	&	Secure Sockets Layer toolkit - cryptographic utility	\\
otg2-oml2	&	2.11.0-mytestbed2	&	amd64	&	Orbit Traffic Generator	\\
paris-traceroute	&	0.92-dev-2	&	amd64	&	New version of well known tool traceroute	\\
passwd	&	1:4.2-3+deb8u1	&	amd64	&	change and administer password and group data	\\
perl	&	5.20.2-3+deb8u4	&	amd64	&	Larry Wall's Practical Extraction and Report Language	\\
perl-base	&	5.20.2-3+deb8u4	&	amd64	&	minimal Perl system	\\
perl-modules	&	5.20.2-3+deb8u4	&	all	&	Core Perl modules	\\
procps	&	2:3.3.9-9	&	amd64	&	/proc file system utilities	\\
python	&	2.7.9-1	&	amd64	&	interactive high-level object-oriented language (default version)	\\
python-cairo	&	1.8.8-1+b2	&	amd64	&	Python bindings for the Cairo vector graphics library	\\
python-dateutil	&	2.2-2	&	all	&	powerful extensions to the standard datetime module	\\
python-glade2	&	2.24.0-4	&	amd64	&	GTK+ bindings: Glade support	\\
python-gobject-2	&	2.28.6-12+b1	&	amd64	&	deprecated static Python bindings for the GObject library	\\
python-gtk2	&	2.24.0-4	&	amd64	&	Python bindings for the GTK+ widget set	\\
python-imaging	&	2.6.1-2+deb8u2	&	all	&	Python Imaging Library compatibility layer	\\
python-lxml	&	3.4.0-1	&	amd64	&	pythonic binding for the libxml2 and libxslt libraries	\\
python-matplotlib	&	1.4.2-3.1	&	amd64	&	Python based plotting system in a style similar to Matlab	\\
python-matplotlib-data	&	1.4.2-3.1	&	all	&	Python based plotting system (data package)	\\
python-meld3	&	1.0.0-1	&	amd64	&	HTML/XML templating system for Python	\\
python-minimal	&	2.7.9-1	&	amd64	&	minimal subset of the Python language (default version)	\\
python-mock	&	1.0.1-3	&	all	&	Mocking and Testing Library	\\
python-nose	&	1.3.4-1	&	all	&	test discovery and running of Python's unittest	\\
python-numpy	&	1:1.8.2-2	&	amd64	&	Numerical Python adds a fast array facility to the Python language	\\
python-pil:amd64	&	2.6.1-2+deb8u2	&	amd64	&	Python Imaging Library (Pillow fork)	\\
python-pkg-resources	&	5.5.1-1	&	all	&	Package Discovery and Resource Access using pkg\_resources	\\
python-pyparsing	&	2.0.3+dfsg1-1	&	all	&	Python parsing module	\\
python-qt4	&	4.11.2+dfsg-1	&	amd64	&	Python bindings for Qt4	\\
python-sip	&	4.16.4+dfsg-1	&	amd64	&	Python/C++ bindings generator runtime library	\\
python-six	&	1.8.0-1	&	all	&	Python 2 and 3 compatibility library (Python 2 interface)	\\
python-support	&	1.0.15	&	all	&	automated rebuilding support for Python modules	\\
python-tk	&	2.7.8-2+b1	&	amd64	&	Tkinter - Writing Tk applications with Python	\\
python-tz	&	2012c+dfsg-0.1	&	all	&	Python version of the Olson timezone database	\\
python-zmq	&	14.4.0-1	&	amd64	&	Python bindings for 0MQ library	\\
python2.7	&	2.7.9-2	&	amd64	&	Interactive high-level object-oriented language (version 2.7)	\\
python2.7-minimal	&	2.7.9-2	&	amd64	&	Minimal subset of the Python language (version 2.7)	\\
qdbus	&	4:4.8.6+git64-g5dc8b2b+dfsg-3+deb8u1	&	amd64	&	Qt 4 D-Bus tool	\\
qtchooser	&	47-gd2b7997-2	&	amd64	&	Wrapper to select between Qt development binary versions	\\
qtcore4-l10n	&	4:4.8.6+git64-g5dc8b2b+dfsg-3+deb8u1	&	all	&	Qt 4 core module translations	\\
readline-common	&	6.3-8	&	all	&	GNU readline and history libraries, common files	\\
rename	&	0.20-3	&	all	&	Perl extension for renaming multiple files	\\
ripwavemon-oml2	&	2.11.0-mytestbed2	&	amd64	&	Report statistics from a Navini RipWave modem	\\
rsync	&	3.1.1-3	&	amd64	&	fast, versatile, remote (and local) file-copying tool	\\
ruby	&	1:2.1.5+deb8u2	&	all	&	Interpreter of object-oriented scripting language Ruby (default version)	\\
ruby2.1	&	2.1.5-2+deb8u2	&	amd64	&	Interpreter of object-oriented scripting language Ruby	\\
rubygems-integration	&	1.8	&	all	&	integration of Debian Ruby packages with Rubygems	\\
sed	&	4.2.2-4+b1	&	amd64	&	The GNU sed stream editor	\\
sensible-utils	&	0.0.9	&	all	&	Utilities for sensible alternative selection	\\
sgml-base	&	1.26+nmu4	&	all	&	SGML infrastructure and SGML catalog file support	\\
shared-mime-info	&	1.3-1	&	amd64	&	FreeDesktop.org shared MIME database and spec	\\
smokeping	&	2.6.9-1+deb8u1	&	all	&	latency logging and graphing system	\\
ssl-cert	&	1.0.35	&	all	&	simple debconf wrapper for OpenSSL	\\
startpar	&	0.59-3	&	amd64	&	run processes in parallel and multiplex their output	\\
supervisor	&	3.0r1-1	&	all	&	A system for controlling process state	\\
systemd	&	215-17+deb8u4	&	amd64	&	system and service manager	\\
systemd-sysv	&	215-17+deb8u4	&	amd64	&	system and service manager - SysV links	\\
sysv-rc	&	2.88dsf-59	&	all	&	System-V-like runlevel change mechanism	\\
sysvinit-utils	&	2.88dsf-59	&	amd64	&	System-V-like utilities	\\
tar	&	1.27.1-2+b1	&	amd64	&	GNU version of the tar archiving utility	\\
tcpd	&	7.6.q-25	&	amd64	&	Wietse Venema's TCP wrapper utilities	\\
tcpdump	&	4.6.2-5+deb8u1	&	amd64	&	command-line network traffic analyzer	\\
tk8.6-blt2.5	&	2.5.3+dfsg-1	&	amd64	&	graphics extension library for Tcl/Tk - library	\\
trace-oml2	&	2.11.0-mytestbed2	&	amd64	&	Measure and record libtrace data using OML	\\
traceroute	&	1:2.0.20-2+b1	&	amd64	&	Traces the route taken by packets over an IPv4/IPv6 network	\\
tzdata	&	2016d-0+deb8u1	&	all	&	time zone and daylight-saving time data	\\
tzdata-java	&	2016d-0+deb8u1	&	all	&	time zone and daylight-saving time data for use by java runtimes	\\
ucf	&	3.0030	&	all	&	Update Configuration File(s): preserve user changes to config files	\\
udev	&	215-17+deb8u4	&	amd64	&	/dev/ and hotplug management daemon	\\
util-linux	&	2.25.2-6	&	amd64	&	Miscellaneous system utilities	\\
x11-common	&	1:7.7+7	&	all	&	X Window System (X.Org) infrastructure	\\
xauth	&	1:1.0.9-1	&	amd64	&	X authentication utility	\\
xdg-user-dirs	&	0.15-2	&	amd64	&	tool to manage well known user directories	\\
xml-core	&	0.13+nmu2	&	all	&	XML infrastructure and XML catalog file support	\\
zlib1g:amd64	&	1:1.2.8.dfsg-2+b1	&	amd64	&	compression library - runtime	\\
	\bottomrule
\end{longtable}
}


%------------------------------------------------------------%
\section{Description of metadata fields}
\label{app:metadataFields}

{\scriptsize
\begin{longtable}{p{3cm}p{12cm}}
	\caption{Field description for metadata topic ``MONROE.META.DEVICE.MODEM''.}\label{tab:metaDeviceModem}\\
	\toprule
	\textbf{Name} & \textbf{Description} \\	\midrule
	\endfirsthead
	\caption{Field description for metadata topic ``MONROE.META.DEVICE.MODEM''. (Continued)}\\
	\toprule
	\textbf{Name} & \textbf{Description} \\	\midrule
	\endhead
	%
	NodeId & Node numerical ID.\\
	Timestamp & Entry timestamp (in milliseconds since UNIX epoch).\\
	DataId & Metadata topic.\\
	DataVersion & Set to \num{1}.\\
	SequenceNumber & Monotonically increasing message counter.\\
	InterfaceName & Name of the interface in the \monroe{} node, e.g., ``sim0'', ``sim1'', ``sim2'', ``eth0'', \ldots\\
	Cid & Cell ID.\\
	DeviceMode & Connection mode of the modem (e.g., 2G, 3G, LTE) indicating the radio access technology the modem uses.\\
	DeviceSubmode & Connection submode for 3G connections (e.g., CDMA, WCDMA, UMTS).\\
	DeviceState & State of the device reported to the network: UNKNOWN (0) - Device state is unknwon; REGISTERED (1) - Device is registered to the network; UNREGISTERED (2) - Device is unregistered from the network; CONNECTED (3) - Device is connected to the network; DISCONNECTED (4) - Device is disconnected from the network.\\
	Ecio & EC/IO, quality/cleanliness of signal from the tower to the modem (dB).\\
	ENodebId & Evolved base station ID.\\
	Iccid & Internationally defined integrated circuit card identifier of the SIM card.\\
	Imsi & Internation Mobile Subscriber Identity.\\
	ImsiMccMnc & Mobile Country Code (MCC) and Mobile Network Code (MNC).\\
	Imei & International Mobile Station Equipment Identity.\\
	IpAddress & IP address assigned to the modem by the operator.\\
	InternalIpAddress & Internal IP address of the modem in the \monroe{} node.\\
	MccMnc & Mobile Country Code (MCC) and Mobile Network Code (MNC).\\
	Operator & Operator name as reported by the network for the interface in which the experiment was run.\\
	Lac & Local Area Code for the current cell (hex).\\
	Rsrp & Reference Signal Received Power (LTE).\\
	Frequency & Frequency in \si{\mega\hertz} (e.g., \num{700}, \num{800}, \num{900}, \num{1800} or \num{2600} in Europe).\\
	Rsrq & Reference Signal Received Quality (valid only for LTE networks). The RSRQ measurement provides additional information when Reference Signal Received Power (RSRP) is not sufficient to make a reliable handover or cell reselection decision. RSRQ considers both the Received Signal Strength Indicator (RSSI) and the number of used Resource Blocks (N) $\mathit{RSRQ} = (N * \mathit{RSRP}) / \mathit{RSSI}$ measured over the same bandwidth.\\
	Band & Band corresponding to the frequency used (e.g., \num{3}, \num{7} or \num{20} in Europe).\\
	Pci & Physical Cell ID.\\
	NwMccMnc & Mobile Country Code (MCC) and Mobile Network Code (MNC) from network (read from the network). The tuple uniquely identifies a mobile network operator (carrier) that is using the GSM (including GSM-R), UMTS, and
	LTE public land mobile networks.\\
	Rscp & Received Signal Code Power (UMTS).\\
	Rssi & Received Signal Strength Indicator.\\
	\bottomrule
\end{longtable}
}

{\scriptsize
	\begin{longtable}{p{3cm}p{12cm}}
		\caption{Field description for metadata topic ``MONROE.META.DEVICE.GPS''.}\label{tab:metaDeviceModem}\\
		\toprule
		\textbf{Name} & \textbf{Description} \\	\midrule
		\endfirsthead
		\caption{Field description for metadata topic ``MONROE.META.DEVICE.GPS''. (Continued)}\\
		\toprule
		\textbf{Name} & \textbf{Description} \\	\midrule
		\endhead
		%
		NodeId & Node numerical ID.\\
		Timestamp & Entry timestamp (in milliseconds since UNIX epoch).\\
		DataId & Metadata topic.\\
		DataVersion & Set to \num{1}.\\
		SequenceNumber & Monotonically increasing message counter.\\		
		Longitude & Decimal degrees (WGS84).\\
		Latitude & Decimal degrees (WGS84).\\
		Altitude & Meters AMSL.\\
		Speed & Speed over ground (knots).\\
		SatelliteCount & Number of satellites being tracked.\\
		Nmea & Raw NMEA string from the GPS receiver.\\
		\bottomrule
	\end{longtable}
}

{\scriptsize
	\begin{longtable}{p{3cm}p{12cm}}
		\caption{Field description for metadata topic ``MONROE.META.CONNECTIVITY''.}\label{tab:metaDeviceModem}\\
		\toprule
		\textbf{Name} & \textbf{Description} \\	\midrule
		\endfirsthead
		\caption{Field description for metadata topic ``MONROE.META.CONNECTIVITY''. (Continued)}\\
		\toprule
		\textbf{Name} & \textbf{Description} \\	\midrule
		\endhead
		%
		NodeId & Node numerical ID.\\
		Timestamp           & Entry timestamp (in milliseconds since UNIX epoch).\\
		DataId              & Metadata topic.\\
		DataVersion         & Set to \num{1}.\\
		SequenceNumber      & Monotonically increasing message counter.\\
		Iccid               & Internationally defined integrated circuit card identifier of the SIM card.\\
		InterfaceName       & Name of the interface in the \monroe{} node, e.g., ``sim0'', ``sim1'', ``sim2'', ``eth0'', \ldots\\
		MccMnc              & Mobile Country Code (MCC) and Mobile Network Code (MNC).\\
		Mode                & The connection mode: UNKNOWN (1),  DISCONNECTED (2),  NO SERVICE (3), 2G (4), 3G (5), LTE (6).\\
		Rssi                & Signal strength.\\
		\bottomrule
	\end{longtable}
}

{\scriptsize
	\begin{longtable}{p{3cm}p{12cm}}
		\caption{Field description for metadata topic ``MONROE.META.NODE.SENSOR''.}\label{tab:metaDeviceModem}\\
		\toprule
		\textbf{Name} & \textbf{Description} \\	\midrule
		\endfirsthead
		\caption{Field description for metadata topic ``MONROE.META.NODE.SENSOR''. (Continued)}\\
		\toprule
		\textbf{Name} & \textbf{Description} \\	\midrule
		\endhead
		%
		NodeId                  & Node numerical ID.\\
		Timestamp               & Entry timestamp (in milliseconds since UNIX epoch).\\
		DataId                  & Metadata topic.\\
		DataVersion             & Set to \num{1}.\\
		SequenceNumber          & Monotonically increasing message counter.\\
		Running                 & Comma separated list of experiment GUIDs.\\		
		Cpu                     & CPU temperature (\si{\degreeCelsius}).\\
%		Modems                  & Modem info service (HTTP Status code).\\
%		Dlb                     & Load balancer info service (HTTP status code).\\
%		UsbMonitor              & USB hub info service (HTTP status code).\\		
		Id                      & Session number (boot counter).\\
		Start                   & Start time (Unix timestamp).\\
		Current                 & Uptime (seconds since start of the session).\\
		Total                   & Uptime (cumulative uptime of the node over all sessions). \\
		Percent                 & Uptime (percent of uptime vs. total lifetime of the node). \\		
		System                  & CPU time spent by the kernel in system activities.\\
		Steal                   & The time that a virtual CPU had runnable tasks, but the virtual CPU itself was not running.\\
		Guest                   & The time spent running a virtual CPU for guest operating systems under the control of the Linux kernel.\\
		IoWait                  & CPU time spent waiting for I/O operations to finish when there is nothing else to do.\\
		Irq                     & CPU time spent handling interrupts.\\
		Nice                    & CPU time spent by nice(1)d programs.\\
		Idle                    & Idle CPU time.\\
		User                    & CPU time spent by normal programs and daemons.\\
		SoftIrq                 & CPU time spent handling ``batched'' interrupts.\\		
		Apps                    & Memory used by user-space applications.\\
		Free                    & Unused memory.\\
		Swap                    & Swap space used.\\		
		usb0                    & Battery level for MiFi at USB0 (0-100, -1 for inactive). \\
		usb0charging            & 1 if USB0 battery is charging, 0 otherwise. \\
		usb1                    & Battery level for MiFi at USB1 (0-100, -1 for inactive). \\
		usb1charging            & 1 if USB1 battery is charging, 0 otherwise. \\
		usb2                    & Battery level for MiFi at USB2 (0-100, -1 for inactive). \\
		usb2charging            & 1 if USB2 battery is charging, 0 otherwise. \\
		\bottomrule
	\end{longtable}
}

{\scriptsize
	\begin{longtable}{p{3cm}p{12cm}}
		\caption{Field description for metadata topic ``MONROE.META.NODE.EVENT''.}\label{tab:metaDeviceModem}\\
		\toprule
		\textbf{Name} & \textbf{Description} \\	\midrule
		\endfirsthead
		\caption{Field description for metadata topic ``MONROE.META.NODE.EVENT''. (Continued)}\\
		\toprule
		\textbf{Name} & \textbf{Description} \\	\midrule
		\endhead
		%
		NodeId              & Node numerical ID.\\
		Timestamp           & Entry timestamp (in milliseconds since UNIX epoch).\\
		DataId              & Metadata topic.\\
		DataVersion         & Set to \num{1}.\\
		SequenceNumber      & Monotonically increasing message counter.\\		
		EventType           & Watchdog.Failed: The system watchdog detected an error symptom.\\
		                    & Watchdog.Repaired: The system watchdog resolved the issue.\\
   		                    & Watchdog.Status: Periodic status messages from the watchdog.\\
   		                    & Maintenance.Start: An interactive login on the node is registered.\\
   		                    & Maintenance.Stop: The interactive login session is closed.\\
   		                    & System.Halt: System halt is requested.\\
   		                    & Scheduling.Started: The node starts to query the scheduling server.\\
		Message             & Extra key for some event types.\\
		User                & Extra key for some event types.\\
		\bottomrule
	\end{longtable}
}

{\scriptsize
	\begin{longtable}{p{3cm}p{12cm}}
		\caption{Field description for metadata topic ``MONROE.EXP.PING''.}\label{tab:metaDeviceModem}\\
		\toprule
		\textbf{Name} & \textbf{Description} \\	\midrule
		\endfirsthead
		\caption{Field description for metadata topic ``MONROE.EXP.PING''. (Continued)}\\
		\toprule
		\textbf{Name} & \textbf{Description} \\	\midrule
		\endhead
		%
		NodeId          & Node numerical ID.\\
		Guid            & Unique experiment identifier.\\
		Timestamp       & Entry timestamp (in milliseconds since UNIX epoch).\\
		SequenceNumber  & Monotonically increasing message counter.\\
		DataId          & Metadata topic.\\
		DataVersion     & Set to \num{1}.\\
		Operator        & Operator name as reported by the network for the interface in which the experiment was run.\\
		Iccid           & Internationally defined integrated circuit card identifier of the SIM card.\\		
		Bytes           & Size of the ping message payload.\\
		Host            & IP of the destination host of the ping probe.\\
		Rtt             & Round-Trip-Time of the ping probe.\\
		\bottomrule
	\end{longtable}
}

{\scriptsize
	\begin{longtable}{p{3cm}p{12cm}}
		\caption{Field description for metadata topic ``MONROE.EXP.HTTP.DOWNLOAD''.}\label{tab:metaDeviceModem}\\
		\toprule
		\textbf{Name} & \textbf{Description} \\	\midrule
		\endfirsthead
		\caption{Field description for metadata topic ``MONROE.EXP.HTTP.DOWNLOAD''. (Continued)}\\
		\toprule
		\textbf{Name} & \textbf{Description} \\	\midrule
		\endhead
		%
		NodeId          & Node numerical ID.\\
		Guid            & Unique experiment identifier.\\
		Timestamp       & Entry timestamp (in milliseconds since UNIX epoch).\\
		SequenceNumber  & Monotonically increasing message counter.\\
		DataId          & Metadata topic.\\
		DataVersion     & Set to \num{1}.\\		
		Operator        & Operator name as reported by the network for the interface in which the experiment was run.\\
		Iccid           & Internationally defined integrated circuit card identifier of the SIM card.\\		
		TotalTime       & Total experiment execution time (in fractional seconds).\\
		Bytes           & Total number of bytes downloaded.\\
		SetupTime       & Time required to set up the HTTP connection.\\
		DownloadTime    & Time spent doing the actual download.\\
		Host            & IP address of the remote host from which data was downloaded.\\
		Speed           & Download speed in bytes/s as measured by the experiment.\\
		Port            & TCP port of the remote host from which data was downloaded.\\
		\bottomrule
	\end{longtable}
}

\end{appendices}


%------------------------------------------------------------%
\section{How to map container folders to Windows paths}
\label{app:mapCotainerPathsWindows}

Before being able to access Windows (host) folders from a container, the drive has to be made available to the containers following these steps:\footnote{\url{https://rominirani.com/docker-on-windows-mounting-host-directories-d96f3f056a2c\#.pdeuy0c4o}}$^{,}$\footnote{Thanks to Lena for pointing to the solution.}
\begin{enumerate}
	\item Access the Docker settings dialog from its taskbar icon:
	\textbf{\begin{center}\includegraphics[width=0.3\textwidth]{mapContainerFolderWindows01.png}\end{center}}
	
	\item From the tab ``Shared Drives'', select the drive you want to make available to the containers, e.g., ``C'':
	\textbf{\begin{center}\includegraphics[width=0.6\textwidth]{mapContainerFolderWindows02.png}\end{center}}
	
	\item You will be prompted for login credentials to access the files:
	\textbf{\begin{center}\includegraphics[width=0.6\textwidth]{mapContainerFolderWindows03.png}\end{center}}
	
	\item Start the container mounting the desired folder:
	{\VerbatimFont\begin{verbatim}$ docker run -v c:/Users/Dell/myresults:/data container_name ls /data\end{verbatim}}
	This command executes \identifier{ls /data} inside an instance of the container ``container\_name,'' after mounting ``\identifier{C:/Users/Dell/myresults/}'' into that path.
	
	\item The folder can be accessed normally from Windows and will reflect changes to any files automatically.
	\textbf{\begin{center}\includegraphics[width=0.6\textwidth]{mapContainerFolderWindows04.png}\end{center}}
\end{enumerate}


%------------------------------------------------------------%

%\bibliographystyle{plain}
%\bibliography{XXX}

%------------------------------------------------------------%

%%% back page with a disclaimer, needed for any public document

\newpage
\makebackpage

\end{document}
